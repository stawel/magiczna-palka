
\chapter*{Wprowadzenie}
\addcontentsline{toc}{chapter}{Wprowadzenie}

W dzisiejszym świecie obserwujemy coraz większe zapotrzebowanie na urządzenia,
które potrafi określić swoje położenie jak i orientację w otaczającej je przestrzeni.
Urządzenia takie mają szerokie zastosowanie w wielu dziedzinach, takich jak
wirtualna rzeczywistość (\textit{Oculus Rift, castAR}), wprowadzanie danych czy 
odtwarzanie trójwymiarowej rzeczywistości na podstawie zdjęć (\textit{VisualSFM}).


W pracy przedstawię prototyp urządzenia, który za pomocą pomiaru odległości od wyznaczonych punktów
z dużą rozdzielczością dochodzących do 0.5mm jest w stanie określić swoje położenie
w przestrzeni jak i swoją orientację. 
Pomiar odległości dokonywany jest za pomocą ultradźwięków o częstotliwości 40Khz.  
Mimo iż metoda jest znana od wielu lat, to z uwagi na relatywnie dużą długości fali ultradźwiękowej,
która dla częstotliwości 40Khz wynosi około 8mm jest do takich celów rzadko stosowana.
W niniejszej pracy udało się to pozorne ograniczenie przezwyciężyć.

Prototyp składa się z \textit{głowicy}, której położenie wraz z orientacją przestrzenną chcemy wyznaczyć
oraz z \textit{aparatu odbiorczego} w kształcie trójkąta wyznaczającego nam stałe punkty odniesienia.
Całość podłączona jest do komputera na którym dokonywane są wszystkie obliczenia.



Przykładowo okulary do wirtualnej rzeczywistości takie jak Oculus Rift [TODO] czy castAR [TODO],
muszą uwzględnić położenie jaki i orientację głowy by na tej podstawie wyświetlić użytkownikowi 
odpowiednią treść. 
Z biegiem lat powstało wiele podejść do tego problemu, do najczęściej stosowanych możemy zaliczyć:
1. wykorzystanie akcelerometrów, żyroskopów i magnetometrów zastosowane w Oculus Rift development kit,
wadą tego rozwiązania jest brak punktów odniesienia (występuje tzw. drift) jak i duży nakład obliczeniowy na
modelowanie zachowania się głowy.
2. wykorzystanie światła strukturalnego i sensorów (zazwyczaj kamery)
 do zebrania obrazu obiektu z liniami struktury projektowanej na niego. Taką metodę wykorzystano w 
 Oculus Rift development kit 2 i castAR, gdzie za źródła światła wykorzystano diody podczerwone umieszczone 
 na okularach dodatkowo przed użytkownikiem umieszczona została kamerę. Wykorzystuje ją również Kinnect [TODO],
który na użytkownika projektuje punktową strukturę światła i na podstawie zniekształceń tej struktury wnioskuje 
 się odległości. Zaletą tego rozwiązania jest stały punkt odniesienia jak i możliwość pomiaru wielu puntów na raz,
 do wad należy zaliczyć stosunkowo niską rozdzielczość jak i duży strumień danych do obróbki.
3. wykorzystanie wielu zdjęć znanego stałego obiektu na postawie, których wyznaczamy pozycję kamery względem niego,
  lub odwrotnie, kamera (lub wiele kamer) jest punktem stałym a wyznaczamy pozycję obiektu.  
 Taką technikę wykorzystano w VidialSFM, jak i w The Pi 3D scanner project, 
 rozwiązanie to cechuje się jednak dość niską rozdzielczością.
 



Należy również wspomnieć, że prototyp nie nadaje się do pomiarów precyzyjnych 
z uwagi na 
dość dużą zależność prędkości dźwięku od temperatury otoczenia,
jednak w pomieszczeniach zamkniętych, w których temperatura otocznia zmienia się nieznacznie
dokładność pomiaru jest zadowalająca.


\newpage
Uwagi: 

Tu należałoby sformułować problem, jaki Pan sobie postawił w pracy i jak Pan podszedł do jego rozwiązywania (ogólnie). 
Należy przedstawić Pana wkład intelektualny.

Gdzieś trzeba też opisać inne znane podobne rozwiązania, bo nie uwierzę, że nie ma żadnych i że jest Pan pionierem w tej dziedzinie.
\newline
Uwagi:(brakuje)

Jakiś dodatek z opisem zawartości dołączonej płyty CD, jeśli zamierza Pan takową do pracy dołączyć.
   