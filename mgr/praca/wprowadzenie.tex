
\chapter*{Wprowadzenie}
\addcontentsline{toc}{chapter}{Wprowadzenie}

W dzisiejszym świecie wyraźnie wzrasta zapotrzebowanie na urządzenia,
które potrafią określić swoje położenie i orientację w przestrzeni.
Urządzenia takie mają szerokie zastosowanie w~wielu dziedzinach, m.in. 
w~wirtualnej rzeczywistości, rozszerzonej rzeczywistości, kartografii lub 
 podczas trójwymiarowego skanowania obiektu.
Przykładowo okulary do wirtualnej rzeczywistości, takie jak \textit{Oculus Rift} \cite{bib:OculusRift} 
czy \textit{castAR} \cite{bib:castAR},
muszą uwzględnić położenie i orientację głowy, aby na tej podstawie wyświetlić użytkownikowi 
odpowiednią treść. 


Powstało wiele rozwiązań tego problemu. Do najczęściej stosowanych możemy zaliczyć:
\begin{enumerate}
 \item 
 Wykorzystanie akcelerometrów, żyroskopów i magnetometrów -- 
takie rozwiązanie zastosowano w \textit{Oculus Rift development kit}. Zaletami tej metody są stosunkowo
prosta konstrukcja i niska cena.
Do wad należy zaliczyć brak stałych punktów odniesienia, co skutkuje występowaniem tzw. dryftu.
\textit{Oculus Rift development kit} radzi sobie z tym problemem, modelując w komputerze możliwe zmiany pozycji głowy.
Jednak rozwiązanie to jest dalekie od idealnego, o czym może świadczyć fakt, że w kolejnej wersji 
urządzenia dodano zewnętrzną kamerę śledzącą pozycję głowy.

\item \label{itm:second_method}
 Projektowanie światła strukturalnego na otoczenie i zbieranie informacji o strukturze 
 światła odbitego za pomocą sensorów, zazwyczaj kamer -- 
 taką metodę wykorzystano w \textit{Microsoft Kinnect} \cite{bib:MicrosoftKinect}.
 Urządzenie wyświetla na otoczenie stały wzór punktów, następnie kamerą na podczerwień
 zbiera informację o zniekształceniu danego wzoru i~na tej podstawie odtwarza  
 trójwymiarową strukturę otoczenia i~położenie urządzenia w~tym otoczeniu.
 Podobną metodę wykorzystuje \textit{Oculus Rift development kit 2} \cite{bib:OculusRiftDK2} oraz 
 \textit{castAR} \cite{bib:castAR} -- tutaj za źródła światła służą diody podczerwone umieszczone na okularach.
 Emitowane przez nie światło rejestruje kamera umieszczona przed użytkownikiem.
 Komputer na podstawie względnego położenia widocznych punktów określa położenie i~orientację
 okularów w przestrzeni.
 Zaletą tego rozwiązania są stałe punkty odniesienia (kamera), a także możliwość pomiaru wielu punktów naraz.
 Wadami są stosunkowo niska rozdzielczość, szczególnie w osi Z, i duży strumień danych do obróbki.

\item
 Wykorzystanie wielu zdjęć, na postawie których 
 wyznaczana jest pozycja kamery względem znajdujących się na nich  stałych (niezmieniających się w czasie) obiektów. 
 Możliwa jest też odwrotna sytuacja, gdy  
  kamera (lub wiele kamer) jest punktem stałym i względem niej wyznaczana jest pozycja fotografowanych obiektów --   
 taką metodę wykorzystano w \textit{VidialSFM} \cite{bib:VisualSFM} oraz w \textit{The Pi 3D scanner project} \cite{bib:pi3dscan}. 
 Wadą tego rozwiązania jest dość niska rozdzielczość.
 
 \item
 Pomiar czasu, jakiego potrzebują fale elektromagnetyczne na dotarcie od nadajników o znanych położeniach 
 do obiektu, którego położenie nas interesuje.
 Dzięki znajomości czasów oraz prędkości rozchodzenia się fal można wyznaczyć odległości pomiędzy nadajnikami a obiektem.
 Na podstawie odległości wyznaczana jest lokalizacja przestrzenna obiektu. 
 Tę metodę powszechnie stosuje się takich systemach jak GPS \cite{bib:gps}, GLONASS \cite{bib:GLONASS} czy Galileo \cite{bib:galileo}.
 Jej zaletą jest możliwość lokalizowania obiektów na dużych przestrzeniach.
 Jednak ze względu na dużą prędkość, z jaką rozchodzą się fale elektromagnetyczne, metoda ta cechuje się dość niską 
 dokładnością nieprzekraczającą \SI{10}{cm}.
 
 \item
 Pomiar siły sygnału radiowego (fal elektromagnetycznych) docierającego do obiektu, którego lokalizacja nas interesuje.
 Źródłem sygnału mogą być dedykowane radiolatarnie, np. \textit{Estimote Beacon} \cite{bib:beacon},
 czy inne nadajniki, np. WiFi \cite{bib:lokWiFi}.
 Wadę tego rozwiązania stanowi niska dokładność pomiaru, która wynosi maksymalnie kilkadziesiąt centymetrów.  
 
 \item
 Wykorzystanie światła podczerwonego naturalnie emitowanego przez badany obiekt --
 takie podejście zaprezentowano w pracy \textit{PIR Sensors: Characterization and Novel Localization Technique}
 \cite{bib:PIRsens}. Ta metoda jest podobna do metody drugiej, z tą różnicą, że zrezygnowano w niej
 ze sztucznego źródła światła, zamiast niego wykorzystując naturalne światło podczerwone emitowane 
 przez badany obiekt. Tutaj również uzyskuje się stosunkowo niską dokładność pomiarową.
 
 \item
 Wykorzystanie indukcji magnetycznej --  metoda polega na pomiarze siły zmiennego pola magnetycznego niskiej częstotliwości
 generowanego przez zewnętrzne anteny. Na podstawie siły pola docierającego do badanego obiektu estymowane są jego odległości
 od anten, a na ich podstawie wyznaczana jest lokalizacja przestrzenna obiektu.
 Główną zaletę tego podejścia stanowi fakt, że pole magnetyczne przenika przez większość materiałów. Pozwala  
 to na lokalizowanie obiektu również pod ziemią, co zaprezentowano w pracy
 \textit{Revealing the hidden lives of underground animals using magneto-inductive tracking} \cite{bib:chomiki}. 
 Wadą metody jest mały zasięg, ponieważ pole magnetyczne propaguje się odwrotnie proporcjonalnie do sześcianu  
 odległości od źródła.
 
 \item 
 Wykorzystanie specyficznych warunków zewnętrznych, w jakich może znajdować się obiekt, którego lokalizacja nas interesuje
 -- w pracy \textit{We Can Track You If You Take the Metro: Tracking Metro
Riders Using Accelerometers on Smartphones} \cite{bib:metro} autorzy opisali algorytm lokalizowania
użytkownika metra na podstawie danych z akcelerometru, które zmieniały się podczas jazdy.
Metoda ta jest jednak dość specyficzna i nie nadaje się do ogólnych zastosowań. 
 
\end{enumerate}
 W niniejszej pracy przedstawiono prototyp, który jest oparty na zmodyfikowanej czwartej metodzie i zamiast 
 fal elektromagnetycznych wykorzystuje ultradźwięki. 
 Podejście to zapewnia dużą dokładność, szczególnie w osi Z, prostotę budowy, a także niską cenę.
 Urządzenie składa się z dwóch części: odbiornika, na którym umieszczono trzy mikrofony, oraz nadajnika,
 na którym znajdują się cztery głośniki. Pomiędzy mikrofonami i głośnikami mierzy się odległość z rozdzielczością
 dochodzącą do \SI{0,5}{mm}, dzięki czemu prototyp jest w stanie określić położenie nadajnika
w przestrzeni i jego orientację. 
Pomiaru odległości dokonuje się za pomocą ultradźwięków o częstotliwości \SI{40}{kHz}.
Mimo iż ta metoda jest znana od wielu lat, to rzadko się ją stosuje do precyzyjnych pomiarów
z uwagi na relatywnie dużą długość fali ultradźwiękowej,
która dla częstotliwości \SI{40}{kHz} wynosi około \SI{8.5}{mm}.

Autorowi niniejszej pracy udało się to pozorne ograniczenie przezwyciężyć.
Ostatecznie urządzenie jest w stanie śledzić położenie nadajnika z rozdzielczością 
\SI{5000}{px} $\times$ \SI{5000}{px} $\times$ \SI{5000}{px} w przestrzeni ograniczonej sześcianem o rozmiarach 
\SI{2,5}{m} $\times$ \SI{2,5}{m}  $\times$ \SI{2,5}{m}, przy czasie odświeżania rzędu \SI{350}{ms}.
Warto podkreślić, że przyjęte parametry sześcianu wynikają ze względów praktycznych i bez 
problemu można je zwiększyć, zachowując stosunek rozdzielczości na metr.
Relatywnie długi czas odświeżania zależy głównie od czasu,  
 jakiego potrzebuje fala dźwiękowa, by się rozproszyć,
 tak by jej odbicia od powierzchni ścian nie wpływały na kolejne pomiary.

Należy nadmienić, że wykorzystanie ultradźwięków do orientacji w terenie jest powszechnie znane,
z powodzeniem wykorzystują je m.in. nietoperze czy niektóre gatunki ryjówek podczas echolokacji. Stosuje się je również 
w przemyśle, głównie do pomiaru odległości, np. w~samochodach podczas parkowania ultradźwiękami
mierzona jest odległości od otaczających je obiektów. Powstało także kilka prac wykorzystujących ultradźwięki
do lokalizacji przestrzennej. 
Przykładowo w pracy \textit{Ultrasonic 3D Wireless Computer Mouse} \cite{bib:mouse} przedstawiono prototyp myszki 3D, której
pozycja w przestrzeni określana jest za pomocą ultradźwięków, niemniej wykorzystany przez autorów algorytm wyznaczania 
odległości jest dużo bardziej podatny na błędy od metody prezentowanej w niniejszej pracy. Ponadto autorzy koncentrują się jedynie
na określaniu położenia myszki w przestrzeni, bez wyznaczania jej orientacji.
Kolejne ciekawe rozwiązanie stanowi rękawica dla graczy 
\textit{Power Glove} \cite{bib:powerGlove}, która pojawiła się w sprzedaży w 1989 roku \cite{bib:powerGlove2}. 
Ona również wykorzystuje ultradźwięki do wyznaczania pozycji
w przestrzeni, jednak nie odniosła  znaczącego sukcesu komercyjnego ze względu na wyjątkowo niską precyzję urządzenia.

Niniejsza praca podzielona została na cztery rozdziały. W pierwszym rozdziale przedstawiono podstawy teoretyczne lokalizacji oraz orientacji 
przestrzennej. W podrozdziale \ref{sec:pomiar} przedstawiono sposób pomiaru odległości za pomocą ultradźwięków, pokazano 
również główny czynnik wpływający na dokładność pomiaru.
Rozdział drugi zawiera opis budowy oraz zasadę działania prezentowanego prototypu. W rozdziale trzecim
przedstawiono zastosowaną metodę przekształcenia zebranych danych ultradźwiękowych na pozycję oraz orientację prototypu.
Rozdział czwarty podsumowuje prezentowane podejście, wyliczając jego zalety, jak i~wady.

