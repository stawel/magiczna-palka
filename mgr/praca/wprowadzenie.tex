
\chapter*{Wprowadzenie}
\addcontentsline{toc}{chapter}{Wprowadzenie}

W dzisiejszym świecie obserwujemy coraz większe zapotrzebowanie na urządzenia,
które potrafią określić swoje położenie jak i orientację w otaczającej je przestrzeni.
Urządzenia takie mają szerokie zastosowanie w wielu dziedzinach m.in. w
wirtualnej rzeczywistości, rozszerzonej rzeczywistości, 
 podczas skanowania trójwymiarowego czy kartografii.
Przykładowo okulary do wirtualnej rzeczywistości takie jak \textit{Oculus Rift} \cite{bib:OculusRift} czy \textit{castAR} \cite{bib:castAR},
muszą uwzględnić położenie jaki i orientację głowy by na tej podstawie wyświetlić użytkownikowi 
odpowiednią treść. 


Z biegiem lat powstało wiele rozwiązań tego problemu, do najczęściej stosowanych możemy zaliczyć:
\begin{enumerate}
 \item 
 wykorzystanie akcelerometrów, żyroskopów i magnetometrów - 
takie rozwiązanie zastosowano w \textit{Oculus Rift development kit}, zaletą metody jest stosunkowo
prosta konstrukcja jak i niska cena,
do wad należy zaliczyć brak stałych punktów odniesienia co skutkuje występowaniem tzw. dryftu.
\textit{Oculus Rift development kit} radzi sobie z tym problemem modelując w komputerze zachowanie się głowy,
jednak rozwiązanie to jest dalekie od idealnego o czym może świadczyć fakt, że w kolejnej wersji 
urządzenia dodano śledzenie głowy przez zewnętrzną kamerę.

\item \label{itm:second_method}
 projektowanie światła strukturalnego na otoczenie i zbieranie informacji o strukturze 
 światła odbitego za pomocą sensorów, zazwyczaj kamer - taką metodę wykorzystano w \textit{Microsoft Kinnect} \cite{bib:MicrosoftKinect},
 urządzenie projektuje na otoczenie stały wzór punktów, następnie kamerą na podczerwień
 zbierana jest informacja o zniekształceniu danego wzoru i na tej podstawie odtwarzana jest 
 trójwymiarowa struktura otoczenia jak i położenie urządzenia w tym otoczeniu.
 Podobną metodę wykorzystuje \textit{Oculus Rift development kit 2} \cite{bib:OculusRiftDK2} jak i 
 \textit{castAR} \cite{bib:castAR}, tutaj za źródła światła służą diody podczerwone umieszczone na okularach,
 światło przez nie emitowane jest rejestrowane poprzez kamerę umieszczoną przed użytkownikiem.
 Komputer na podstawie względnego położenia widocznych punktów określa położenie i orientację
 okularów w przestrzeni.
 Zaletą tego rozwiązania są stałe punkty odniesienia (kamera) jak i możliwość pomiaru wielu puntów na raz.
 Do wad należy zaliczyć stosunkowo niską rozdzielczość szczególnie w osi Z jak i duży strumień danych do obróbki.

\item
 wykorzystanie wielu zdjęć zawierających stałe (nie zmieniające się w czasie) obiekty, na postawie których wyznaczana jest pozycja kamery względem nich
  lub odwrotnie, kamera (lub wiele kamer) jest punktem stałym, a wyznaczana jest pozycja fotografowanych obiektów -   
 taką metodę wykorzystano w \textit{VidialSFM} \cite{bib:VisualSFM}, jak i w \textit{The Pi 3D scanner project} \cite{bib:pi3dscan}, 
 rozwiązanie to cechuje się również dość niską rozdzielczością.
 
\end{enumerate}
 

 W niniejszej pracy przedstawiono prototyp oparty na zmodyfikowanej metodzie drugiej, który zamiast światła wykorzystuje ultradźwięki. 
 Podejście to zapewnia dużo większą dokładność szczególnie w osi Z, prostotę budowy jak i dużo niższą cenę.
 Urządzenie składa się z dwóch części: odbiornika, na którym umieszczone są trzy mikrofony, i nadajnika,
 na którym znajdują się cztery głośniki. Pomiędzy nimi dokonywany jest pomiar odległości z rozdzielczością
 dochodzącą do \SI{0,5}{mm} dzięki czemu prototyp jest w stanie określić położenie nadajnika
w przestrzeni jak i jego orientację. 
Pomiar odległości dokonywany jest za pomocą ultradźwięków o częstotliwości \SI{40}{kHz},
mimo iż metoda jest znana od wielu lat, to z uwagi na relatywnie dużą długość fali ultradźwiękowej,
która dla częstotliwości \SI{40}{kHz} wynosi około \SI{8}{mm} jest do precyzyjnych pomiarów rzadko stosowana.
W niniejszej pracy udało się to pozorne ograniczenie przezwyciężyć.
Ostatecznie urządzenie jest w stanie śledzić położenie nadajnika z rozdzielczością 
\SI{5000}{px} $\times$ \SI{5000}{px} $\times$ \SI{5000}{px} na przestrzeni sześcianu o rozmiarach 
\SI{2,5}{m} $\times$ \SI{2,5}{m}  $\times$ \SI{2,5}{m} przy czasie odświeżania rzędu \SI{350}{ms}.
Warto podkreślić, że wielkość sześcianu została ograniczona ze względów praktycznych i bez 
większego problemu można ją zwiększyć zachowując stosunek rozdzielczości na metr.
Jeśli chodzi o stosunkowo długi czas odświeżania to
 głównym czynnikiem na niego wpływającym jest czas jaki potrzebuje fala dźwiękowa by się rozproszyć,
 tak by jej odbicia od powierzchni ścian nie wpływały na kolejne pomiary.

Należy również wspomnieć o innych pracach, które w podobny sposób podchodzą do problemu pozycjonowania, przykładowo 
w pracy \textit{Ultrasonic 3D Wireless Computer Mouse} \cite{bib:mouse} przedstawiono prototyp myszki 3D, której
pozycja w przestrzeni wyznaczana jest za pomocą ultradźwięków, niemniej wykorzystany przez autorów algorytm wyznaczania 
odległości jest dużo bardziej podatny na błędy. Dodatkowo praca koncentruje się jedynie
na określaniu położenia myszki w przestrzeni, bez wyznaczania jej orientacji.
Kolejnym ciekawym rozwiązaniem jest rękawica dla graczy 
\textit{Power Glove} \cite{bib:powerGlove} \cite{bib:powerGlove2} dla
\textit{Nintendo Entertainment System} wydana w roku 1989, ona również wykorzystuje ultradźwięki do wyznaczania pozycji
rękawicy w przestrzeni, jednak precyzja urządzenia jest wyjątkowo niska przez co rękawica nie odniosła znaczącego sukcesu komercyjnego.


