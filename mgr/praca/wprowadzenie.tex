
\chapter*{Wprowadzenie}
\addcontentsline{toc}{chapter}{Wprowadzenie}

W dzisiejszym świecie obserwujemy coraz większe zapotrzebowanie na urządzenia,
które potrafią określić swoje położenie jak i orientację w otaczającej je przestrzeni.
Urządzenia takie mają szerokie zastosowanie w wielu dziedzinach, takich jak
wirtualna rzeczywistość, wprowadzanie danych czy odtwarzanie trójwymiarowej rzeczywistości na podstawie zdjęć.


Przykładowo okulary do wirtualnej rzeczywistości takie jak \textit{Oculus Rift} [TODO] czy \textit{castAR} [TODO],
muszą uwzględnić położenie jaki i orientację głowy by na tej podstawie wyświetlić użytkownikowi 
odpowiednią treść. 
Z biegiem lat powstało wiele rozwiązań tego problemu, do najczęściej stosowanych możemy zaliczyć:
\begin{enumerate}
 \item 
 wykorzystanie akcelerometrów, żyroskopów i magnetometrów - 
takie rozwiązanie zastosowano w \textit{Oculus Rift development kit}, zaletą takiego rozwiązania jest stosunkowo
prosta konstrukcja jak i niska cena,
do wad należy zaliczyć brak stałych punktów odniesienia co skutkuje występowaniem tzw. dryftu.
\textit{Oculus Rift development kit} radzi sobie z tym problemem modelując w komputerze zachowanie się głowy,
jednak rozwiązanie to jest dalekie od idealnego o czym może świadczyć fakt, że w kolejnej wersji 
urządzenia porzucono to podejście.

\item
 wykorzystanie światła strukturalnego i sensorów (zazwyczaj kamer)
 do zbierania obrazów obiektu z liniami struktury projektowanej na ów obiekt. Taką metodę wykorzystano w 
 Oculus Rift development kit 2 i castAR, gdzie za źródła światła wykorzystano diody podczerwone umieszczone 
 na okularach dodatkowo przed użytkownikiem umieszczona została kamerę. Wykorzystuje ją również Kinnect [TODO],
który na użytkownika projektuje punktową strukturę światła i na podstawie zniekształceń tej struktury wnioskuje 
 się odległości. Zaletą tego rozwiązania jest stały punkt odniesienia jak i możliwość pomiaru wielu puntów na raz,
 do wad należy zaliczyć stosunkowo niską rozdzielczość jak i duży strumień danych do obróbki.

\end{enumerate}


3. wykorzystanie wielu zdjęć znanego stałego obiektu na postawie, których wyznaczamy pozycję kamery względem niego,
  lub odwrotnie, kamera (lub wiele kamer) jest punktem stałym a wyznaczamy pozycję obiektu.  
 Taką technikę wykorzystano w VidialSFM, jak i w The Pi 3D scanner project, 
 rozwiązanie to cechuje się jednak dość niską rozdzielczością.
 

 W pracy przedstawię prototyp urządzenia, który za pomocą pomiaru odległości od wyznaczonych punktów
z dużą rozdzielczością dochodzącą do 0.5mm jest w stanie określić swoje położenie
w przestrzeni jak i swoją orientację. 
Pomiar odległości dokonywany jest za pomocą ultradźwięków,
mimo iż metoda jest znana od wielu lat, to z uwagi na relatywnie dużą długości fali ultradźwiękowej,
która dla częstotliwości 40kHz wynosi około 8mm jest do precyzyjnych pomiarów rzadko stosowana.
W niniejszej pracy udało się to pozorne ograniczenie przezwyciężyć.

Prototyp składa się z \textit{głowicy}, której położenie wraz z orientacją przestrzenną chcemy wyznaczyć
oraz z \textit{aparatu odbiorczego} w kształcie trójkąta wyznaczającego nam stałe punkty odniesienia.
Całość podłączona jest do komputera na którym dokonywane są wszystkie obliczenia.



Należy również wspomnieć, że prototyp nie nadaje się do pomiarów precyzyjnych 
z uwagi na 
dość dużą zależność prędkości dźwięku od temperatury otoczenia,
jednak w pomieszczeniach zamkniętych, w których temperatura otocznia zmienia się nieznacznie
dokładność pomiaru jest zadowalająca.


\newpage
Uwagi: 

Tu należałoby sformułować problem, jaki Pan sobie postawił w pracy i jak Pan podszedł do jego rozwiązywania (ogólnie). 
Należy przedstawić Pana wkład intelektualny.

Gdzieś trzeba też opisać inne znane podobne rozwiązania, bo nie uwierzę, że nie ma żadnych i że jest Pan pionierem w tej dziedzinie.
\newline
Uwagi:(brakuje)

Jakiś dodatek z opisem zawartości dołączonej płyty CD, jeśli zamierza Pan takową do pracy dołączyć.
   