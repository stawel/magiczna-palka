
\chapter*{Wprowadzenie}
\addcontentsline{toc}{chapter}{Wprowadzenie}

W dzisiejszym świecie obserwujemy coraz większe zapotrzebowanie na urządzenia,
które potrafią określić swoje położenie jak i orientację w otaczającej je przestrzeni.
Urządzenia takie mają szerokie zastosowanie w wielu dziedzinach m.in. w
wirtualnej rzeczywistości, rozszerzonej rzeczywistości, 
 podczas skanowania trójwymiarowego czy kartografii.
Przykładowo okulary do wirtualnej rzeczywistości takie jak \textit{Oculus Rift} \cite{bib:OculusRift} czy \textit{castAR} \cite{bib:castAR},
muszą uwzględnić położenie jaki i orientację głowy by na tej podstawie wyświetlić użytkownikowi 
odpowiednią treść. 


Z biegiem lat powstało wiele rozwiązań tego problemu, do najczęściej stosowanych możemy zaliczyć:
\begin{enumerate}
 \item 
 wykorzystanie akcelerometrów, żyroskopów i magnetometrów - 
takie rozwiązanie zastosowano w \textit{Oculus Rift development kit}, zaletą metody jest stosunkowo
prosta konstrukcja jak i niska cena,
do wad należy zaliczyć brak stałych punktów odniesienia co skutkuje występowaniem tzw. dryftu.
\textit{Oculus Rift development kit} radzi sobie z tym problemem modelując w komputerze zachowanie się głowy,
jednak rozwiązanie to jest dalekie od idealnego o czym może świadczyć fakt, że w kolejnej wersji 
urządzenia dodano śledzenie głowy przez zewnętrzną kamerę.

\item \label{itm:second_method}
 projektowanie światła strukturalnego na otoczenie i zbieranie informacji o strukturze 
 światła odbitego za pomocą sensorów, zazwyczaj kamer - taką metodę wykorzystano w \textit{Microsoft Kinnect} \cite{bib:MicrosoftKinect},
 urządzenie projektuje na otoczenie stały wzór punktów, następnie kamerą na podczerwień
 zbierana jest informacja o zniekształceniu danego wzoru i na tej podstawie odtwarzana jest 
 trójwymiarowa struktura otoczenia jak i położenie urządzenia w otoczeniu.
 Podobną metodę wykorzystuje \textit{Oculus Rift development kit 2} \cite{bib:OculusRiftDK2} jak i 
 \textit{castAR} \cite{bib:castAR}, w tym przypadku za źródła światła służą diody podczerwone umieszczone na okularach, które
 rejestrowane są bezpośrednio przez kamerę umieszczoną przed użytkownikiem.
 Program na podstawie względnego położenia widocznych punktów światła na obrazie określa położenie i orientację
 okularów w przestrzeni.
 Zaletą tego rozwiązania są stałe punkty odniesienia jak i możliwość pomiaru wielu puntów na raz.
 Do wad należy zaliczyć stosunkowo niską rozdzielczość szczególnie w osi $Z$ jak i duży strumień danych do obróbki.

\item
 wykorzystanie wielu zdjęć znanych stałych obiektów, na postawie których wyznaczamy pozycję kamery względem nich
  lub odwrotnie, kamera (lub wiele kamer) jest punktem stałym a wyznaczamy pozycję obiektów -   
 taką metodę wykorzystano w \textit{VidialSFM} \cite{bib:VisualSFM}, jak i w \textit{The Pi 3D scanner project} \cite{bib:pi3dscan}, 
 rozwiązanie to cechuje się jednak dość niską rozdzielczością.
 
\end{enumerate}
 

 W pracy przedstawiono prototyp oparty na zmodyfikowanej \ref{itm:second_method}
 metodzie, który zamiast światła wykorzystuje ultradźwięki. Podejście to zapewnia dużo większą
 dokładność w szczególności w osi $Z$, prostotę budowy jak i dużo niższą cenę.
 Urządzenie za pomocą pomiaru odległości od wyznaczonych punktów
z dużą rozdzielczością dochodzącą do \SI{0,5}{mm} jest w stanie określić swoje położenie
w przestrzeni jak i swoją orientację. 
Pomiar odległości dokonywany jest za pomocą ultradźwięków o częstotliwości \SI{40}{kHz},
mimo iż metoda jest znana od wielu lat, to z uwagi na relatywnie dużą długości fali ultradźwiękowej,
która dla częstotliwości \SI{40}{kHz} wynosi około \SI{8}{mm} jest do precyzyjnych pomiarów rzadko stosowana.
W niniejszej pracy udało się to pozorne ograniczenie przezwyciężyć,
ostatecznie urządzenie jest w stanie śledzić położenie głowicy z rozdzielczością 
$\SI{5000}{px} \times \SI{5000}{px} \times \SI{5000}{px}$ na przestrzeni sześcianu o rozmiarach 
$\SI{2,5}{m} \times \SI{2,5}{m}  \times \SI{2,5}{m} $ przy czasie odświeżania rzędu \SI{350}{ms}.
Warto podkreślić, że wielkość sześcianu została ograniczona ze względów praktycznych i bez 
większego problemu można ją zwiększyć zachowując stosunek rozdzielczości na metr.
Głównym czynnikiem jaki wpływa na czas odświeżania jest czas jaki fala dźwiękowa potrzebuje
by się rozproszyć, tak by jej odbicia nie wpływały na kolejne pomiary.

Przedstawiany prototyp składa się z dwóch części: \textit{głowicy}, której położenie wraz z orientacją przestrzenną chcemy wyznaczyć
oraz z \textit{aparatu odbiorczego} w kształcie trójkąta wyznaczającego stałe punkty odniesienia.
Całość  podłączona jest do komputera za pomocą złącza USB, na którym dokonywane są wszystkie obliczenia,
końcowym wynikiem jest trójwymiarowy widok, na którym zaznaczona jest pozycja głowicy oraz dwa wektory normalne
przedstawiające orientacje głowicy.




   