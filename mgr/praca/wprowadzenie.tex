
\chapter*{Wprowadzenie}
\addcontentsline{toc}{chapter}{Wprowadzenie}

W dzisiejszym świecie wyraźnie wzrasta zapotrzebowanie na urządzenia,
które potrafią określić swoje położenie i orientację w otaczającej je przestrzeni.
Urządzenia takie mają szerokie zastosowanie w wielu dziedzinach, m.in. w
wirtualnej rzeczywistości, rozszerzonej rzeczywistości, 
 podczas skanowania trójwymiarowego obiektów czy kartografii.
Przykładowo okulary do wirtualnej rzeczywistości takie jak \textit{Oculus Rift} \cite{bib:OculusRift} 
czy \textit{castAR} \cite{bib:castAR}
muszą uwzględnić położenie i orientację głowy, aby na tej podstawie wyświetlić użytkownikowi 
odpowiednią treść. 


Z biegiem lat powstało wiele rozwiązań tego problemu. Do najczęściej stosowanych możemy zaliczyć:
\begin{enumerate}
 \item 
 Wykorzystanie akcelerometrów, żyroskopów i magnetometrów -- 
takie rozwiązanie zastosowano w \textit{Oculus Rift development kit}. Zalety tej metody to stosunkowo
prosta konstrukcja i niska cena.
Do wad należy zaliczyć brak stałych punktów odniesienia, co skutkuje występowaniem tzw. dryftu.
\textit{Oculus Rift development kit} radzi sobie z tym problemem, modelując w komputerze możliwe zmiany pozycji głowy.
Jednak rozwiązanie to jest dalekie od idealnego, o czym może świadczyć fakt, że w kolejnej wersji 
urządzenia dodano zewnętrzną kamerę śledzącą pozycję głowy.

\item \label{itm:second_method}
 Projektowanie światła strukturalnego na otoczenie i zbieranie informacji o strukturze 
 światła odbitego za pomocą sensorów, zazwyczaj kamer -- 
 taką metodę wykorzystano w \textit{Microsoft Kinnect} \cite{bib:MicrosoftKinect}.
 Urządzenie projektuje na otoczenie stały wzór punktów, następnie kamerą na podczerwień
 zbiera informację o zniekształceniu danego wzoru i na tej podstawie odtwarza  
 trójwymiarową strukturę otoczenia i położenie urządzenia w tym otoczeniu.
 Podobną metodę wykorzystuje \textit{Oculus Rift development kit 2} \cite{bib:OculusRiftDK2} oraz 
 \textit{castAR} \cite{bib:castAR} -- tutaj za źródła światła służą diody podczerwone umieszczone na okularach.
 Emitowane przez nie światło jest rejestrowane przez kamerę umieszczoną przed użytkownikiem.
 Komputer na podstawie względnego położenia widocznych punktów określa położenie i orientację
 okularów w przestrzeni.
 Zaletą tego rozwiązania są stałe punkty odniesienia (kamera), a także możliwość pomiaru wielu puntów naraz.
 Wady stanowią stosunkowo niska rozdzielczość, szczególnie w osi Z, jak i duży strumień danych do obróbki.

\item
 Wykorzystanie wielu zdjęć, na postawie których 
 wyznaczana jest pozycja kamery względem znajdujących się na nich  stałych (niezmieniających się w czasie) obiektów. 
 Możliwa jest też odwrotna sytuacja, gdy  
  kamera (lub wiele kamer) jest punktem stałym i względem niej wyznaczana jest pozycja fotografowanych obiektów --   
 taką metodę wykorzystano w \textit{VidialSFM} \cite{bib:VisualSFM} oraz w \textit{The Pi 3D scanner project} \cite{bib:pi3dscan}. 
 Wadą tego rozwiązanie jest dość niska rozdzielczość.
 
\end{enumerate}
 

 W niniejszej pracy przedstawiono prototyp oparty na zmodyfikowanej metodzie drugiej, który zamiast światła wykorzystuje ultradźwięki. 
 Podejście to zapewnia dużo większą dokładność, szczególnie w osi Z, prostotę budowy, a także niższą cenę.
 Urządzenie składa się z dwóch części: odbiornika, na którym umieszczone są trzy mikrofony, i nadajnika,
 na którym znajdują się cztery głośniki. Pomiędzy nimi dokonywany jest pomiar odległości z rozdzielczością
 dochodzącą do \SI{0,5}{mm}, dzięki czemu prototyp jest w stanie określić położenie nadajnika
w przestrzeni i jego orientację. 
Pomiar odległości dokonuje się za pomocą ultradźwięków o częstotliwości \SI{40}{kHz}.
Mimo iż ta metoda jest znana od wielu lat, to jest  rzadko stosowana do precyzyjnych pomiarów
z uwagi na relatywnie dużą długość fali ultradźwiękowej,
która dla częstotliwości \SI{40}{kHz} wynosi około \SI{8}{mm}.

Autorowi niniejszej pracy udało się to pozorne ograniczenie przezwyciężyć.
Ostatecznie urządzenie jest w stanie śledzić położenie nadajnika z rozdzielczością 
\SI{5000}{px} $\times$ \SI{5000}{px} $\times$ \SI{5000}{px} w przestrzeni ograniczonej sześcianem o rozmiarach 
\SI{2,5}{m} $\times$ \SI{2,5}{m}  $\times$ \SI{2,5}{m}, przy czasie odświeżania rzędu \SI{350}{ms}.
Warto podkreślić, że przyjęte parametry sześcianu wynikają ze względów praktycznych i bez 
problemu można je zwiększyć, zachowując stosunek rozdzielczości na metr.
Stosunkowo długi czas odświeżania zależy głównie od czasu,  
 jaki potrzebuje fala dźwiękowa, by się rozproszyć,
 tak by jej odbicia od powierzchni ścian nie wpływały na kolejne pomiary.

Należy również wspomnieć o innych pracach, które w podobny sposób podchodzą do problemu pozycjonowania. Przykładowo 
w pracy \textit{Ultrasonic 3D Wireless Computer Mouse} \cite{bib:mouse} przedstawiono prototyp myszki 3D, której
pozycja w przestrzeni określana jest za pomocą ultradźwięków, niemniej wykorzystany przez autorów algorytm wyznaczania 
odległości jest dużo bardziej podatny na błędy. Dodatkowo praca ta koncentruje się jedynie
na określaniu położenia myszki w przestrzeni, bez wyznaczania jej orientacji.
Kolejne ciekawe rozwiązanie stanowi rękawica dla graczy 
\textit{Power Glove} \cite{bib:powerGlove} dla
\textit{Nintendo Entertainment System}, która pojawiła się w sprzedaży w  1989 roku \cite{bib:powerGlove2}. 
Ona również wykorzystuje ultradźwięki do wyznaczania pozycji
w przestrzeni, jednak nie odniosła  znaczącego sukcesu komercyjnego ze względu na wyjątkowo niską precyzję urządzenia.


