\chapter{przetwarzanie danych zebranych z odbiorników}

Przetwarzaniem danych z odbiorników zajmuje się biblioteka \textit{mp3d}
oraz program \textit{oscy} wizualizujący dane w postaci obrazu trójwymiarowego.

Biblioteka \textit{mp3d} została podzielona na pięć modułów:
\begin{enumerate}

 \item \textit{com.py} - moduł odpowiedzialny za komunikację z głównym odbiornikiem
 \item \textit{find\_pattern.py} - moduł odpowiedzialny za wyszukanie wzorca w odebranym sygnale
 \item \textit{xyz.py} - moduł odpowiedzialny za wyznaczenie pozycji i orientacji głowicy, oraz za 
 weryfikację, czy zebrane dane odpowiadają rzeczywistości
 \item \textit{info.py} - moduł wyświetlający informację o sile odbieranego sygnału
 \item \textit{ply.py} - moduł odpowiedzialny za eksportowanie danych do formaty \textit{.ply} 
    (Polygon File Format) obsługiwanego przez większość programów do obróbki grafiki 3D.

\end{enumerate}


\section{moduł \textit{com.py}}

Zadaniem modułu \textit{com.py} jest komunikacja poprzez port USB z głównym odbiornikiem,
moduł pracuje w oddzielnym wątku, w którym cyklicznie uruchamiany jest jeden z nadajników umieszczonych na głowicy,
następnie sczytuje zebrane sygnały z trzech mikrofonów umieszczonych na głównym odbiorniku.
Ta czynność powtarzana jest dla każdego z czterech nadajników. 
Zebrane dwanaście sygnałów po wstępnej filtracji przekazywane są dalej do \textit{find\_pattern.py}. Cały cykl powtarzany jest
co 200ms. Rysunek \ref{fig:com_output_2m} przedstawia sygnał przekazywany do modułu \textit{find\_pattern.py}.


\begin{figure}[h!]
    \centering
    \includegraphics[width=1.15\textwidth, trim= 47mm 0mm 0mm 0mm,clip]{com_output_2m_1}
    \includegraphics[width=1.15\textwidth, trim= 47mm 0mm 0mm 0mm,clip]{com_output_2m_2}
    \caption{sygnał odebrany przez moduł \textit{com.py}. 
    Odległość między nadajnikiem a odbiornikiem wynosi 2 metry.
    Na górnym wykresie po prawej stronie widoczne są również sygnały odbite od ścian.
    }
    \label{fig:com_output_2m}
\end{figure}


\section{moduł \textit{find\_pattern.py}}

Głównym modułem biblioteki \textit{mp3d} jest moduł \textit{find\_pattern.py}.
Odpowiada on za znalezienie \textit{wzorca} w odebranym sygnale z odbiorników,

Niech $w(t)$  dla $t = 0..n-1$ będzie szukanym wzorcem, a $f(x)$ odebranym sygnałem,
wtedy możemy znaleźć takie $a$, że błąd średniokwadratowy pomiędzy $w(t)$ i $a f(t+x)$ jest minimalny:
\[
  e(x) = \min_{a \in R} \{ \sum_{t=0}^{n-1}  (w(t) - a f(t+x))^2 \}
\]
zauważmy, że:
\[
  e(x) = \min_{a \in R} \{ \sum_{t=0}^{n-1}  (w^2(t) -2a w(t) f(t+x) + a^2 f^2(t+x)) \}
\]
\[
  e(x) = \min_{a \in R} \{ \sum_{t=0}^{n-1}  w^2(t) -2a \sum_{t=0}^{n-1}  w(t) f(t+x) + a^2 \sum_{t=0}^{n-1} f^2(t+x) \}
\]
funkcja kwadratowa w postaci:
\[
  y(a) = \sum_{t=0}^{n-1}  w^2(t) -2a \sum_{t=0}^{n-1}  w(t) f(t+x) + a^2 \sum_{t=0}^{n-1} f^2(t+x)
\]
osiąga minimum dla:
\[
 a = \frac{ \sum\limits_{t=0}^{n-1}  w(t) f(t+x) }{ \sum\limits_{t=0}^{n-1} f^2(t+x) }
\]
z czego ostatecznie dostajemy:

\[
  e(x) = \sum_{t=0}^{n-1}  w^2(t)  - \frac {(\sum\limits_{t=0}^{n-1}  w(t) f(t+x) )^2 } { \sum\limits_{t=0}^{n-1} f^2(t+x)}
\]

Zauważmy, że dzięki skalowaniu sygnału $f(x)$ przez parametr $a$ zamiast sygnał wzorcowy $w(x)$
otrzymany błąd $e(x)$ nie zależy od siły sygnału, co ułatwia porównanie błędów w dwóch różnych miejscach.
Ponadto wyliczenie $ \sum\limits_{t=0}^{n-1}  w^2(t) $ 
jak i $\sum\limits_{t=0}^{n-1} f^2(t+x)$ wymaga jedynie liniowej liczby operacji, a 
 $\sum\limits_{t=0}^{n-1}  w(t) f(t+x)  $ jest korelacją wzajemną funkcji $w(t)$ i $f(x)$, którą
 można wyliczyć w czasie $n \log(n)$ korzystając z szybkiej transformacji Fouriera \cite{bib:FFT_correlation}.
 
 Funkcję $e(x)$  można ją interpretować jako:
 im mniejszy błąd $e(x)$ tym większe prawdopodobieństwo, że szukany wzorzec $w$ znajduje się na pozycji $x$ w 
 odebranym sygnale $f$. 
 Wynikiem modułu \textit{find\_pattern.py} jest cała funkcja, $e(x)$ na podstawie której moduł \textit{xyz.py}
 wyznaczy pozycję głowicy w przestrzeni jak i jego orientację uwzględniając przy tym 
 kształt głowicy (nadmiarowość danych) jak i prawdopodobieństwa że znaleziony wzorzec jest na danej pozycji.
 
 Kolejną zadaniem modułu \textit{find\_pattern.py} jest 