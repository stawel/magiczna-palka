\chapter{biblioteka mp3d}


W ramach projekty powstała również biblioteka \textit{mp3d}, której zadaniem jest analiza zebranych danych,
oraz wyznaczenie położenia i orientacji głowicy prototypu.
Biblioteka została podzielona na pięć modułów:
\begin{enumerate}

 \item \textit{com.py} - moduł odpowiedzialny za komunikację z głównym odbiornikiem
 \item \textit{find\_pattern.py} - moduł odpowiedzialny za wyszukanie wzorca w odebranym sygnale
 \item \textit{xyz.py} - moduł odpowiedzialny za wyznaczenie pozycji i orientacji głowicy, oraz za 
 weryfikację, czy zebrane dane odpowiadają rzeczywistości
 \item \textit{info.py} - moduł wyświetlający informację o sile odbieranego sygnału
 \item \textit{ply.py} - moduł odpowiedzialny za eksportowanie danych do formaty \textit{.ply} 
    (Polygon File Format) obsługiwanego przez większość programów do obróbki grafiki 3D.

\end{enumerate}


\section{moduł \textit{com.py}}

Zadaniem modułu \textit{com.py} jest komunikacja poprzez port USB z głównym odbiornikiem,
ponieważ w pytonie komunikacja jest synchroniczna, moduł pracuje w oddzielnym wątku, w którym cyklicznie 
uruchamia jeden z nadajników umieszczonych na głowicy,
następnie sczytuje zebrane sygnały z trzech mikrofonów umieszczonych na głównym odbiorniku.
Tą czynność powtarza dla każdego z czterech nadajników, zebrane dwanaście sygnałów 
po wstępnej filtracji przekazywane są dalej do obróbki. Cały cykl powtarzany jest
co 200ms.

\section{moduł \textit{find\_pattern.py}}

Moduł \textit{find\_pattern.py} odpowiedzialny jest za wyszukanie wzorca dla 
