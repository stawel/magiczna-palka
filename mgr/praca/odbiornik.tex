\section{Odbiornik}

Centralną część prezentowanego prototypu stanowi odbiornik.
Jego zadaniem jest wysyłanie sygnałów sterujących do nadajnika, zbieranie ultradźwięków z otoczenia oraz przesyłanie 
ich do komputera w celu dalszej analizy.
Odbiornik składa się z następujący części (rysunek \ref{fig:odbiornik_szkic}):

\begin{itemize}
 \item trzech modułów ultradźwiękowych przetwarzających dźwięk na sygnał elektryczny,
 \item płytki prototypowej \textit{stm32f4-discovery} \cite{bib:stm32f4Discovery} odpowiadającej za komunikację z komputerem,
 \item przystawki do \textit{stm32f4-discovery} przystosowującej sygnały elektryczne z modułów ultradźwiękowych
  do poziomów akceptowalnych przez tę płytkę,
 \item ramy, na której umieszczono moduły ultradźwiękowe.
\end{itemize}


\rysunek{odbiornik_szkic}{Szkic odbiornika}{\label{fig:odbiornik_szkic}}


\section{Budowa i zasada działania odbiornika}

Głównym elementem odbiornika jest płytka prototypowa \textit{stm32f4-discovery} \cite{bib:stm32f4Discovery}.
 Umożliwia ona komunikację wszystkich komponentów z komputerem.
\textit{Stm32f4-discovery} wyposażona jest w procesor STM32F407VGT6 \cite{bib:stm32f407} (oparty o rdzeń ARM Cortex M4), 
który zawiera trzy 12-bitowe przetworniki
analogowo-cyfrowe umożliwiające próbkowanie z prędkością do \SI{2,4}{MSPS} 
(ang. \textit{MSPS -- Mega-Samples Per Second} -- megapróbek na sekundę). Przetworniki te wykorzystane zostały do próbkowania
sygnałów pochodzących z modułów ultradźwiękowych. Płytka umożliwia również komunikację z komputerem przez 
port USB z prędkością \SI{12}{MB/s}. Znajduje się na także programator
umożliwiający programowanie procesora za pośrednictwem dodatkowego portu USB.

Na potrzeby odbiornika powstało dedykowane oprogramowanie w C sterujące procesorem.
Zostało ono oparte na bibliotece \textit{stm32 usb 101} \cite{bib:stm32_usb_101} zapewniającej komunikację z komputerem. 
Do biblioteki dodano obsługę przetworników analogowo-cyfrowych oraz możliwość sterowania nadajnikiem.
Program przez port USB dostaje instrukcję, który z czterech głośników ma nadawać, i przekazuje ją
dalej do nadajnika wraz z sygnałem wyzwalającym. Następnie uruchamiane są równocześnie trzy przetworniki analogowo-cyfrowe, które 
próbkują odbierany dźwięk i poprzez DMA zapisują wynik w pamięci procesora.
Częstotliwość pracy przetworników ustawiono na \SI{1,6}{MSPS}, co daje 40 próbek na jeden okres sygnału o częstotliwości \SI{40}{kHz}.
Po zebraniu w sumie $3\times\SI{16}{k}=\SI{48}{k}$ próbek całość przesyłana jest do komputera w celu dalszej analizy.
Proces powtarza się dla każdego z czterech głośników nadajnika, 
co w sumie daje 12 sygnałów, na podstawie których wyznaczona zostaje 
pozycja w przestrzeni oraz orientacja nadajnika.

Całą elektronikę osadzono na ramie w kształcie trójkąta zbudowanej z rur PCV  (rysunek \ref{fig:trojkat}). 
Odległości pomiędzy modułami ultradźwiękowymi są z góry ustalone, co ułatwia dalsze obliczenia.

Warto zaznaczyć, że dla dźwięku poruszającego się z prędkością \SI{340}{m/s} długość wykorzystanej 
fali dźwiękowej wynosi \SI{8.5}{mm}, co przy 
pomiarze 40 próbek na jeden okres daje rozdzielczość pomiaru odległości mniejszą niż \SI{0.5}{mm} (dokładnie \SI{0.2125}{mm}).
Błąd pomiaru trudno jednak oszacować, ponieważ  zależy on od aktualnej prędkości
dźwięku, a ta zmienia się w czasie wraz ze zmianą warunków atmosferycznych.

 \begin{figure}[h]
    \centering
    \includegraphics[width=0.8\textwidth, trim= 0mm 20mm 0mm 0mm,clip]{trojkat}
    \caption{Szkic ramy odbiornika}
    \label{fig:trojkat}
\end{figure}


\section{Budowa modułu ultradźwiękowego}

Odbiornik wyposażono w trzy moduły ultradźwiękowe, których zadaniem jest 
zbieranie ultradźwięków z trzech różnych  punktów.
Każdy moduł zawiera przetwornik piezoelektryczny (mikrofon) 40SR-12 \cite{bib:40ST12},
który przetwarza sygnał akustyczny na odpowiadający mu sygnał elektryczny, oraz wzmacniacze operacyjne 
wstępnie zwiększające amplitudę sygnału, który przesyłany jest dalej do przystawki.
Rysunek \ref{fig:odbiornik_ultra} przedstawia schemat modułu ultradźwiękowego.

\rysunek{receiver}{Schemat modułu ultradźwiękowego}{\label{fig:odbiornik_ultra}}

Wzmacniacz operacyjny IC1A wraz z kondensatorem C2 i rezystorem R1 pracuje 
jako przedwzmacniacz ładunkowy \cite{bib:wzm_ladunkowy}.
Ładunek wytworzony na przetworniku piezoelektrycznym SP1 zostaje w całości przeniesiony na kondensator C2 
(wzmacniacz utrzymuje różnicę potencjałów między dodatnim a ujemnym wejściem na zerowym poziomie).
W rezultacie na kondensatorze pojawia się napięcie zgodnie z równaniem $U=\frac{q}{C}$.
Do rozładowania kondensatora C2 służy rezystor R1.
R1 i C2 działają również jako filtr dolnoprzepustowy.

Wzmacniacz IC1B z rezystorami R5 i R4 oraz kondensatorami C3, C7 i C8 pracuje jako wzmacniacz napięciowy wzbogacony o 
filtr pasmowoprzepustowy.
Sygnał z IC1B za pośrednictwem wtyczki JP1 dociera do przystawki współpracującej z \textit{stm32f4-discovery}.

W celu zminimalizowania zakłóceń zastosowano niskoszumowe wzmacniacze operacyjne
mieszczące się w jednym układzie scalonym NE5532 \cite{bib:ne5532}. 
Dodatkowo płytka drukowana jest ekranowana.


\section{Przystawka do \textit{stm32f4-discovery}}

Sygnał z modułów ultradźwiękowych dociera do \textit{stm32f4-discovery} za pośrednictwem specjalnej przystawki.
Schemat budowy przystawki przedstawia rysunek \ref{fig:przystawka}.
Przystawka przystosowuje maksymalne amplitudy zebranych sygnałów do wartości akceptowalnych przez  
przetworniki analogowo-cyfrowe procesora STM32F407VGT6.
Wartości te muszą mieścić się w zakresie od \SI{0}{V} do \SI{3,3}{V}.

Zastosowano w niej układ TLV2774 \cite{bib:TLV2774}, który zawiera 4 wzmacniacze operacyjne typu
\textit{rail-to-rail}. Trzy z nich wykorzystano jako ostatni stopień wzmocnienia sygnałów ultradźwiękowych. 
Wzmacniacze operacyjne pracują w układzie odwracającym, którego wzmocnienie można regulować potencjometrami R8, R9 oraz R10. 
Podłączono je do wspólnego, również regulowanego (przy użyciu potencjometru R7) napięcia odniesienia.
Przystawka zawiera także stabilizator napięcia LM78M05CDT \cite{bib:LM78M05CDT}, który po podłączeniu 
do JP4 baterii \SI{12}{V} dostarcza zasilanie do wszystkich komponentów. 
Istnieje możliwość odcięcia zasilania przez rozwarcie JP5, co jest konieczne podczas programowania
STM32F407VGT6, by zapobiec uszkodzeniu stabilizatora napięcia.
Z przystawki przez wtyczkę JP6 wyprowadzono również sygnały sterujące nadajnikiem oraz zasilanie.

 \begin{figure}[h]
    \centering
    \includegraphics[width=1\textwidth, trim= 5mm 0mm 0mm 0mm,clip]{mainboard2}
    \caption{Schemat budowy przystawki do \textit{stm32f4-discovery}}
    \label{fig:przystawka}
\end{figure}


