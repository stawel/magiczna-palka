\chapter{Odbiornik}

Odbiornik ultradźwiękowy składa się z 6 części:
płytki prototypowej \textit{stm32f4-discovery} zbierającej dane analizowane później na komputerze,
 trzech odbiorników ultradźwiękowych wraz ze wzmacniaczami, 
 \textit{shieldu} do \textit{stm32f4-discovery} - przystawki dopasowującej sygnał z odbiorników do poziomów akceptowalnych przez
\textit{stm32f4-discovery}, oraz ramy na której umieszczone są odbiorniki.

\section{Budowa i zasada działania}

Głównym elementem odbiornika stanowi płytka prototypowa \textit{stm32f4-discovery} \cite{bib:stm32f4Discovery},
jej zadaniem jest przesłanie dźwięków zebranych z otoczenia do dalszej analizy oraz wysłanie sygnału wyzwalającego 
do nadajnika.
\textit{Stm32f4-discovery} oparty jest o procesor STM32F407VGT6 \cite{bib:stm32f407}, który 
umożliwia komunikację z komputerem za pomocą złącza USB oraz ma wbudowane trzy 12-bitowe przetworniki analogowo-cyfrowe
mogące samplować z częstotliwością do 2.4MSPS. 


Oprogramowanie płytki prototypowej oparte jest na bibliotece \textit{stm32 usb 101} \cite{bib:stm32_usb_101}
zapewniającej komunikację z komputerem, do której dodana została obsługa przetworników ADC.
Program poprzez port USB dostaje komunikat, który z czterech nadajników ma nadawać, ta informacja przekazywana jest
dalej do nadajnika w raz z sygnałem wyzwalającym, następnie uruchamiane są trzy przetworniki ADC, które 
samplują odbierany dźwięk i poprzez DMA zapisują trzy kanały w pamięci procesora.
Częstotliwość pracy przetworników ustawiona jest na 1.6MSPS co daje średnio 40 sampli na jedną sinusoidę nadanego 40kHz sygnału.
Program zapamiętuje 16kS na każdym kanale, co przy prędkości dźwięku 340m/s daje maksymalną mierzoną odległość rzędu 3.5 metrów.
Po zebraniu w sumie 48kS, całość przesyłana jest do komputera w celu dalszej analizy.
Cały proces powtarzany jest dla każdego z czterech przetworników piezoelektrycznych, 
co w sumie daje 12 sygnałów na podstawie których wyznaczona zostaje 
pozycja w przestrzeni oraz orientacja nadajnika.

Cała elektronika osadzona została na ramie zbudowanej z rur PCV w kształcie trójkąta (rysunek \ref{fig:trojkat}), 
Odległości pomiędzy odbiornikami są z góry znane, co ułatwia obliczenia.

\rysunek{trojkat}{szkic ramy odbiornika}{\label{fig:trojkat}}



\clearpage
\section{Budowa odbiornika ultradźwiękowego}

Fale dźwiękowe przetwarzane są na sygnał elektryczny za pomocą odbiornika ultradźwiękowego (piezoelektrycznego) 
podłączonego do dwóch wzmacniaczy operacyjnych. Schemat przedstawiony jest na rysunku \ref{fig:odbiornik_ultra}.

\rysunek{receiver}{schemat odbiornika ultradźwiękowego}{\label{fig:odbiornik_ultra}}

Pierwszy ze wzmacniaczy pracuje jako przedwzmacniacz ładunkowy \cite{bib:wzm_ladunkowy}.
Ładunek wytworzony na przetworniku piezoelektrycznym zostaje w całości przeniesiony na kondensator $C2$ 
(wzmacniacz stara się utrzymać różnicę potencjałów miedzy dodatnim a ujemnym wejściem na zerowym poziomie),
co jest równoważne z pojawieniem się napięcia na kondensatorze zgodnie z równaniem $U=\frac{q}{C}$.
Rezystor $R1$ ustawia napięcia spoczynkowe układu na poziome $\frac{1}{2}$ Vcc jak i rozładowuje kondensator $C2$.
$R1$ i $C2$ działają również jako filtr górnoprzepustowy.

Drugi wzmacniacz pracuje jako zwykły wzmacniacz napięciowy wzbogacony o filtr górno i dolnoprzepustowy.

Aby zminimalizować zakłócenia całość oparta jest na niskoszumowym wzmacniaczu operacyjnym NE5532 \cite{bib:ne5532}, 
dodatkowo cała płytka $PCB$ jest ekranowana.

Wzmocniony sygnał poprzez wtyczkę $JP1$ doprowadzony jest do \textit{shieldu} współpracującego z \textit{stm32f4-discovery}.

\clearpage

\section{\textit{Shield} do \textit{stm32f4-discovery}}

Sygnał z odbiorników doprowadzany jest do \textit{stm32f4-discovery} za pomocą specjalnej przystawki (\textit{shieldu}).
Jej zadaniem jest przystosowanie amplitud zebranych sygnałów do wartości akceptowalnych przez STM32F407VGT6,
dodatkowo przystawka zawiera układ zasilający jak i wyprowadzone są z niej sygnały sterujące do nadajnika.
Zakres napięć dopuszczalnych dla przetowrnika analogowo-cyfrowego zawierają się w przedziale od 0V do 3.3V,
dlatego zastosowany został układ TLV2774 \cite{bib:TLV2774}, który zawiera w sobie 4 wzmacniacze operacyjne typu
rail-to-rail. Wzmacniacze pracują w układzie odwracającym, do którego dodano regulowane napięcie odniesienia.
Schemat przystawki przedstawiony jest na rysunku \ref{fig:shild}.



\rysunek{mainboard2}{schemat \textit{shieldu} do \textit{stm32f4-discovery}}{\label{fig:shild}}


