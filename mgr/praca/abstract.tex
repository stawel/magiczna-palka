\begin{abstract}
W pracy przedstawiono prototyp urządzenia potrafiącego określić zarówno swoje
położenie, jak i orientację w przestrzeni. Urządzenie składa się z dwóch części:
nadajnika, którego pozycja i orientacja jest wyznaczana, oraz odbiornika, który 
stanowi stały punkt odniesienia. Wyznaczanie położenia odbywa się za 
pomocą pomiarów odległości pomiędzy nadajnikiem a odbiornikiem -- wykorzystuje  się
do tego ultradźwięki o częstotliwości \SI{40}{kHz}. 
Zastosowana metoda umożliwia pomiar z wysoką rozdzielczością dochodzącą do \SI{0.5}{mm}.
\end{abstract}
