\chapter{Podsumowanie}

W niniejszej pracy przedstawiono urządzenie, które za pomocą ultradźwięków  potrafi określić zarówno swoje
położenie, jak i orientację w przestrzeni.

Mimo że przy pomiarze położenia wykorzystywane są fale dźwiękowe o długość \SI{8}{mm}, urządzenie to
potrafi osiągnąć rozdzielczość  rzędu \SI{0,5}{mm}, co znacząco wyróżnia je na tle innych rozwiązań.
Posiada także inne zalety, jak również wady, które opisano poniżej.
\newline
Zalety:
\begin{itemize}
 \item Wysoka rozdzielczość pomiaru -- zastosowanie częstotliwości 
 próbkowania dużo wyższych od częstotliwości nadawanego sygnału, a także wyszukiwanie wzorca pozwoliły
 wielokrotnie zwiększyć rozdzielczość pomiaru w stosunku do długości wykorzystanej fali dźwiękowej.
 
 \item Praktycznie nieograniczona przestrzeń robocza -- mimo że w prezentowanym prototypie
 ograniczono przestrzeń roboczą do sześcianu \SI{2,5}{m} $\times$ \SI{2,5}{m} $\times$ \SI{2,5}{m},
 jest to jedynie ograniczenie praktyczne i bez trudu można dowolnie zwiększyć tę przestrzeń. Trzeba jednak 
 pamiętać, że siła sygnału dźwiękowego maleje kwadratowo wraz z odległością, więc przy dużo większych odległościach
 niezbędne staje się zastosowanie nadajników większej mocy. 
 
 \item Niski koszt komponentów.
\end{itemize}
Wady:
\begin{itemize}
 \item Duże gabaryty nadajnika i~odbiornika.
 
 \item Duża czułość na zmiany temperatury -- jest to najpoważniejsza wada prezentowanego rozwiązania.
 Zauważmy, że zmiana temperatury otoczenia o \ang{1}\SI{}{C} powoduje zmianę prędkości dźwięku o 0,17\%,
 co przy rozdzielczości \SI{5000}{px} w skrajnych przypadkach daje przesunięcie \SI{7.5}{px}.
 Co prawda w warunkach laboratoryjnych nie stanowi to większego problemu, jednak w praktycznych zastosowaniach
 jest poważnym ograniczeniem.
 
 \item Długi czas odświeżania -- prezentowany prototyp dokonuje pomiarów co \SI{350}{ms}, czyli około 3 pomiarów 
 na sekundę. Ponieważ mierzymy cztery niezależne sygnały (z czterech głośników), poszczególne sygnały mierzone są co \SI{87}{ms}.
  Dźwięk poruszający się z~prędkością \SI{346}{m/s}  przebędzie  w tym czasie odległość \SI{29.7}{m}.
 Tak duży dystans w~stosunku do powierzchni roboczej jest konieczny, by zapobiec nakładaniu się sygnałów
 (które po odbiciu się od ścian mogłyby wpłynąć na kolejne pomiary; czas \SI{87}{ms} pozwala 
 na ich całkowite rozproszenie).
 Zakłócaniu pomiarów można także zaradzić, nadając za każdym razem inny rozróżnialny sygnał, to podejście 
 wymaga jednak nadajników i~odbiorników o~większych możliwościach wysterowania.
 
 \item Stosunkowo mały kąt widzenia -- zastosowane przetworniki piezoelektryczne są przetwornikami kierunkowymi, co 
 ogranicza  maksymalny kąt, pod jakim można ustawić nadajnik względem odbiornika. W prezentowanym 
 prototypie kąt ten nie powinien przekraczać \ang{45}. Zastosowanie większej liczby przetworników lub innego ich rodzaju
 pozwoliłoby znacznie zwiększyć kąt.

 \item Ograniczone możliwości wysterowania rezonatorów piezoelektrycznych -- 
 zastosowane rezonatory nie nadają się do generowania krótkich impulsów, co zdecydowanie uprościłoby mierzenie odległości.
 
 \item Problem z ustaleniem pozycji i~orientacji nadajnika w ruchu -- w prezentowanym urządzeniu 
 jeden pomiar pozycji i orientacji wymaga określania położenia czterech głośników w różnych odstępach czasu.
 Jeśli podczas takiego pomiaru nadajnik porusza się ze zbyt dużą prędkością, to otrzymamy położenia
 przesunięte względem siebie zgodnie z kierunkiem ruchu nadajnika. Takie przesunięcie negatywnie wpływa na 
 pomiar orientacji. 
\end{itemize}


Reasumując, prezentowane urządzenie, wykorzystujące ultradźwięki do określania pozycji i orientacji,
jest innowacyjną propozycją, która zapewnia znacznie większą dokładność niż wszystkie inne dostępne rozwiązania.
Najistotniejszą jego wadą wydaje się jednak duża wrażliwość na warunki atmosferyczne, a szczególnie na zmianę temperatury.
Jest to zapewne główny powód, dlaczego rozwiązanie to nie znajdzie  szerokiego zastosowania w produktach komercyjnych.



