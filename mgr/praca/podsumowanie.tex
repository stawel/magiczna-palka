\chapter{Podsumowanie}

W niniejszej pracy zademonstrowano urządzenie potrafiące określić swoje
położenie jak i orientację w przestrzeni wykorzystują do tego celu ultradźwięki.
Mimo że wykorzystano fale dźwiękowe o długość \SI{8}{mm}, pomiar położenia osiąga rozdzielczości rzędu \SI{0,5}{mm} 
co znacząco wyróżnia się na tle innych rozwiązań.
W tym rozdziale przedstawiono zalety jaki i wady prezentowanego rozwiązania.

Zalety:
\begin{itemize}
 \item Wysoka rozdzielczość pomiaru. Zastosowanie dużo wyższych częstotliwości 
 próbkowania od częstotliwości nadawanego sygnału, jak i zastosowane wyszukiwanie wzorca pozwoliło
 wielokrotnie zwiększyć rozdzielczość pomiaru w stosunku to długości wykorzystanej fali dźwiękowej.
 
 \item Praktycznie nieograniczona przestrzeń robocza. Mimo, że w prezentowanym prototypie
 ograniczono przestrzeń roboczą do kwadratu \SI{2,5}{m} $\times$ \SI{2,5}{m} $\times$ \SI{2,5}{m},
 to jest to jedynie ograniczenie praktyczne i bez większych problemów można ją dowolnie zwiększyć. Trzeba jednak 
 pamiętać, że siła sygnału dźwiękowego maleje kwadratowo wraz z odległością więc przy dużo większych odległościach
 niezbędne będą nadajniki większej mocy. 
 
 \item Niski koszt budowy. 
\end{itemize}

Wady:
\begin{itemize}
 \item Duża czułość na zmiany temperatury. Jest to najpoważniejsza wada prezentowanego rozwiązania.
 Zauważmy, że zmiana temperatury otoczenia rzędu \SI{1}{\degC} powoduje zmianę prędkości dźwięku o \SI{0,17}{\%} 
 co przy rozdzielczości \SI{5000}{px} w skrajnych przypadkach daje przesunięcie \SI{7.5}{px},
 co prawda w warunkach laboratoryjnych nie stanowi to większego problemu jednak w praktycznych zastosowaniach
 jest poważnym ograniczeniem.
 
 \item Długi czas odświeżania. Prezentowany prototyp dokonuje pomiary co \SI{350}{ms}, czyli około 3 pomiarów 
 na sekundę. Zauważmy, że mierząc cztery niezależne sygnały (z czterech głośników) kolejny sygnał mierzony jest co \SI{87}{ms},
  dźwięk poruszający się z prędkością \SI{346}{m/s} w tym czasie przebędzie odległość 29.7 metrów.
 Tak duży dystans w stosunku do powierzchni roboczej jest konieczny by zapobiec nakładaniu się sygnałów pochodzących z poprzednich
 pomiarów z nowymi (nadany sygnał odbija się od ścian i wpływa na kolejne pomiary, czas rzędu \SI{87}{ms} pozwala 
 na jego całkowite rozproszenie).
 Można temu zaradzić nadając za każdym razem inny rozróżnialny sygnał, to podejście 
 wymaga jednak nadajników i odbiorników o większych możliwościach wysterowania.
 
 \item Stosunkowo mały kąt widzenia. Zastosowane przetworniki piezoelektryczne są przetwornikami kierunkowymi, co 
 znacząco wpływa na maksymalny kąt pod jakim można ustawić nadajnik względem odbiornika. W prezentowanym 
 prototypie kąt ten nie powinien przekraczać \ang{45}. Zastosowanie większej liczby przetworników lub inny ich rodzaj
 powinien znacząco zwiększyć ów kąt.

 \item Ograniczone możliwości wysterowania rezonatorów piezoelektrycznych. 
 Zastosowane rezonatory nie nadają się do generowania krótkich impulsów, co znacząco uprościłyby mierzenie odległości.
 
 \item Problem z pomiarem pozycji i orientacji nadajnika w ruchu. W prezentowanym urządzeniu 
 jeden pomiar pozycji i orientacji wymaga pomiaru położenia czterech głośników w różnych odstępach czasowym.
 Jeśli podczas takiego pomiaru nadajnik porusza się z większą prędkością to otrzymane położenia
 przesunięte są względem siebie zgodnie z kierunkiem ruchu nadajnika. Takie przesunięcie negatywnie wpływa na 
 pomiar orientacji. 
\end{itemize}


Jak widać, wykorzystanie dźwięków to określania pozycji i orientacji prócz niewątpliwych zalet ma również
wiele wad. Najistotniejszą wadą wydaje się być duża wrażliwość na warunki atmosferyczne a szczególnie zmianę temperatury.
Jest to zapewne główny powód dlaczego ów rozwiązanie nie jest szeroko stosowane w produktach komercyjnych.



