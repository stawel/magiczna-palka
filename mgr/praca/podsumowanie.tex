\chapter{Podsumowanie}

W niniejszej pracy przedstawiono urządzenie, które za pomocą ultradźwięków  potrafi określić zarówno swoje
położenie, jak i orientację w przestrzeni.

Mimo że przy pomiarze położenia wykorzystywane są fale dźwiękowe o długość \SI{8}{mm}, urządzenie to
potrafi osiągnąć rozdzielczość  rzędu \SI{0,5}{mm}, co znacząco wyróżnia je na tle innych rozwiązań.
Posiada także inne zalety, jak również wady, które opisano poniżej.
\newline
Zalety:
\begin{itemize}
 \item Wysoka rozdzielczość pomiaru. Zastosowanie częstotliwości 
 próbkowania dużo wyższych od częstotliwości nadawanego sygnału, a także wyszukiwanie wzorca pozwoliły
 wielokrotnie zwiększyć rozdzielczość pomiaru w stosunku do długości wykorzystanej fali dźwiękowej.
 
 \item Praktycznie nieograniczona przestrzeń robocza. Mimo że w prezentowanym prototypie
 ograniczono przestrzeń roboczą do sześcianu \SI{2,5}{m} $\times$ \SI{2,5}{m} $\times$ \SI{2,5}{m},
 jest to jedynie ograniczenie praktyczne i bez trudu można dowolnie zwiększyć tę przestrzeń. Trzeba jednak 
 pamiętać, że siła sygnału dźwiękowego maleje kwadratowo wraz z odległością, więc przy dużo większych odległościach
 niezbędne staje się zastosowanie nadajniki większej mocy. 
 
 \item Niski koszt komponentów.
\end{itemize}

Wady:
\begin{itemize}
 \item Duża czułość na zmiany temperatury. Jest to najpoważniejsza wada prezentowanego rozwiązania.
 Zauważmy, że zmiana temperatury otoczenia o \SI{1}{\degC} powoduje zmianę prędkości dźwięku o \SI{0,17}{\%},
 co przy rozdzielczości \SI{5000}{px} w skrajnych przypadkach daje przesunięcie \SI{7.5}{px}.
 Co prawda w warunkach laboratoryjnych nie stanowi to większego problemu, jednak w praktycznych zastosowaniach
 jest poważnym ograniczeniem.
 
 \item Długi czas odświeżania. Prezentowany prototyp dokonuje pomiarów co \SI{350}{ms}, czyli około 3 pomiarów 
 na sekundę. Ponieważ mierzymy cztery niezależne sygnały (z czterech głośników), poszczególne sygnały mierzone są co \SI{87}{ms}.
  Dźwięk poruszający się z prędkością \SI{346}{m/s}  przebędzie  w tym czasie odległość 29.7 metrów.
 Tak duży dystans w stosunku do powierzchni roboczej jest konieczny, by zapobiec nakładaniu się sygnałów
 (które po odbiciu się od ścian mogłyby wpłynąć na kolejne pomiary; czas rzędu \SI{87}{ms} pozwala 
 na ich całkowite rozproszenie).
 Zakłócaniu pomiarów można także zaradzić, nadając za każdym razem inny rozróżnialny sygnał, to podejście 
 wymaga jednak nadajników i odbiorników o większych możliwościach wysterowania.
 
 \item Stosunkowo mały kąt widzenia. Zastosowane przetworniki piezoelektryczne są przetwornikami kierunkowymi, co 
 ogranicza  maksymalny kąt, pod jakim można ustawić nadajnik względem odbiornika. W prezentowanym 
 prototypie kąt ten nie powinien przekraczać \ang{45}. Zastosowanie większej liczby przetworników lub innego ich rodzaju
 pozwoliłoby znacznie zwiększyć te kąt.

 \item Ograniczone możliwości wysterowania rezonatorów piezoelektrycznych. 
 Zastosowane rezonatory nie nadają się do generowania krótkich impulsów, co zdecydowanie uprościłoby mierzenie odległości.
 
 \item Problem z ustaleniem pozycji i orientacji nadajnika w ruchu. W prezentowanym urządzeniu 
 jeden pomiar pozycji i orientacji wymaga pomiaru położenia czterech głośników w różnych odstępach czasu.
 Jeśli podczas takiego pomiaru nadajnik porusza się ze zbyt dużą prędkością, to otrzymamy położenie
 przesunięte względem siebie zgodnie z kierunkiem ruchu nadajnika. Takie przesunięcie negatywnie wpływa na 
 pomiar orientacji. 
\end{itemize}


Reasumując: prezentowane urządzenie, wykorzystujące ultradźwięki do określania pozycji i orientacji,
jest innowacyjną propozycją, która zapewnia znacznie większą dokładność niż wszystkie inne dostępne rozwiązania.
Najistotniejszą jego wadą wydaje się jednak duża wrażliwość na warunki atmosferyczne, a szczególnie zmianę temperatury.
Jest to zapewne główny powód, dlaczego rozwiązanie to nie znajdzie  szerokiego zastosowania w produktach komercyjnych.



