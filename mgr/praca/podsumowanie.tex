\chapter{Podsumowanie}

W niniejszej pracy zademonstrowano urządzenie potrafiące określić swoje
położenie i orientację wykorzystują do tego celu ultradźwięki.
Mimo że wykorzystane fale mają długość równą \SI{8}{mm}  bez większego problemu
udało się mierzyć położenie z rozdzielczością rzędu \SI{0,5}{mm} co dla 
pozycjonowania przestrzennego jest dużym osiągnięciem.
Mimo wielu zalet prezentowane rozwiązanie posiada również szereg wad, w niniejszym rozdziale
przedstawione zostaną za i przeciw zastosowanego rozwiązania.

Zalety:
\begin{itemize}
 \item Wysoka rozdzielczość pomiaru. Zastosowanie dużo wyższych częstotliwości 
 próbkowania od częstotliwości nadawanego sygnału, jak i zastosowane wyszukiwanie wzorca pozwoliło
 wielokrotnie zwiększyć rozdzielczość pomiaru w stosunku to długości wykorzystanej fali dźwiękowej.
 
 \item Praktycznie nieograniczona przestrzeń pracy. Mimo, że prezentowanym prototypie
 ograniczono przestrzeń roboczą do kwadratu \SI{2,5}{m} $\times$ \SI{2,5}{m} $\times$ \SI{2,5}{m},
 to jest to jedynie ograniczenie praktyczne i bez większych problemów można ją dowolnie zwiększyć. Trzeba jednak 
 pamiętać, że siła sygnały dźwiękowego maleje kwadratowo z odległością więc przy dużo większych odległościach
 niezbędny będzie nadajnik większej mocy. Większa powierzchnia pomiarowa wpływa również na czas odświeżania.

 \item Niski koszt budowy. 
\end{itemize}

Wady:
\begin{itemize}
 \item Duża czułość na zmiany temperatury. Jest to najpoważniejsza wada prezentowanego rozwiązania.
 Zauważmy, że zmiana temperatury otoczenia rzędu \SI{1}{\degC} powoduje zmianę prędkości dźwięku o \SI{0,17}{\%} 
 co przy rozdzielczości \SI{5000}{px} w skrajnych przypadkach daje przesunięcie \SI{7.5}{px},
 co prawda w warunkach laboratoryjnych nie stanowi to większego problemu jednak w praktycznych zastosowaniach
 jest poważnym ograniczeniem.
 
 \item Długi czas odświeżania. Prezentowany prototyp dokonuje pomiary co \SI{350}{ms}, co daje około 3 pomiarów 
 na sekundę. Zauważmy, że mierząc cztery niezależne sygnały (z czterech głośników) każdy sygnał mierzony jest co \SI{87}{ms},
 co dla dźwięku poruszającego się z prędkością \SI{346}{m/s} pozwala na przebycie dystansu 29.7 metrów.
 Tak duży dystans w stosunku do powierzchni roboczej jest konieczny by zapobiec nakładaniu się sygnałów pochodzących z poprzednich
 pomiarów z nowymi (sygnał odbija się od ścian i wpływa na kolejne pomiary, czas rzędu \SI{87}{ms} pozwala 
 na całkowite rozproszenie sygnału).
 Można temu zaradzić nadając za każdym razem inny sygnał, następnie poszukiwany wzorzec należy odfiltrować z odebranego
 sygnału, to podejście nie jest jednak możliwe gdy za odbiorniki i nadajniki wykorzystujemy rezonatory piezoelektryczne
 (możliwości ich wysterowania są mocno ograniczone).
 
 \item Stosunkowo mały kąt widzenia. Zastosowane nadajniki piezoelektryczne są nadajnikami kierunkowymi, co 
 znacząco wpływa na maksymalny kąt pod jakim można ustawić nadajnik względem odbiornika. W prezentowanym 
 prototypie kąt ten nie powinien przekraczać \SI{45}{\degree}. Zastosowanie większej liczby nadajników lub inny ich rodzaj
 może znacząco zwiększyć ten kąt.

 \item [TODO] Ograniczone możliwości wysterowania rezonatorów piezoelektrycznych. 
 Zastosowane rezonatory nie nadają się do generowania krótkich impulsów, które znacząco uprościłyby mierzenie odległości.
 Dużo lepszym rozwiązaniem powinny być standardowe mikrofony i głośniki ultradźwiękowe, niestety są dużo droższe 
 i trudno dostępne.
 
 \item Problem z mierzeniem pozycji i orientacji w ruchu. Zauważmy, że w prezentowanym rozwiązaniu wyznaczane są odległości od
 kilku puntów, a pomiary są dokonywany w pewnych odstępach czasu. Jeśli mierzony obiekt się porusza to pomiar odległości w rożnych
 momentach w czasie powoduje przekłamania w pozycjonowaniu przestrzennym.
 
\end{itemize}

Jak widać, wykorzystanie dźwięków to określania pozycji i orientacji ma zdecydowanie więcej wad niż zalet,
szczególnym problemem okazała się wrażliwość prototypu na zmiany temperatury, jest on na tyle wrażliwy, że 
sama temperatura ciała operatora urządzenia jak i jego oddech wpływa na pomiar odległości, oczywiście tym przypadku błędy pomiarowe
nie są duże, rzędu ~\SI{1}{mm} ale są zauważalne.
