\documentclass{pracamgr}
\usepackage{polski}
\usepackage{amssymb}
\usepackage{amsthm}
\usepackage{amsmath}
\usepackage{algorithmic}
\usepackage{algorithm}
\usepackage{pdfpages}
\usepackage{sistyle}
\usepackage[utf8]{inputenc}
\usepackage[pdfborder={0 0 0}]{hyperref}

\SIdecimalsign{,}
%include ps
\usepackage{graphicx}
%\epstopdfsetup{update}
%\DeclareGraphicsExtensions{.ps}
%\epstopdfDeclareGraphicsRule{.ps}{pdf}{.pdf}{ps2pdf -dEPSCrop -dNOSAFER #1 \OutputFile}


% Dane magistranta:

\author{Paweł Stawicki}
\nralbumu{189254}

\title{Wyznaczanie pozycji oraz orientacji w przestrzeni za pomocą ultradźwięków}
\tytulang{Determining position and orientation in space using ultrasound}

\kierunek{Informatyka}
\opiekun{dra inż. Marcina Peczarskiego}

\date{Wrzesień 2015}

%Podać dziedzinę wg klasyfikacji Socrates-Erasmus:
\dziedzina{ 
%11.0 Matematyka, Informatyka:\\ 
%11.1 Matematyka\\ 
%11.2 Statystyka\\ 
11.3 Informatyka\\ 
%11.4 Sztuczna inteligencja\\ 
%11.5 Nauki aktuarialne\\
%11.9 Inne nauki matematyczne i informatyczne
}

%Klasyfikacja tematyczna wedlug AMS (matematyka) lub ACM (informatyka)
%\klasyfikacja{D. Software\\
%  D.127. Blabalgorithms\\
%  D.127.6. Numerical blabalysis}
\klasyfikacja{
TODO: dodać
%I.4.8 Scene Analysis, Subject: Tracking
%B.0 [Hardware]: General
%B.4 [Hardware]: Input/Output & Data Communication 
%H.4.m [Information Systems Applications]: Mis-
%cellaneous
}

% S?owa kluczowe:
\keywords{położenie, lokalizacja, orientacja, 3D, ultradźwięki}

\renewcommand{\algorithmicrequire}{\textbf{Input:}}
\renewcommand{\algorithmicensure}{\textbf{Output:}}
\renewcommand{\algorithmiccomment}[1]{// #1}

% Tu jest dobre miejsce na Twoje w?asne makra i ?rodowiska:
\newtheorem{defi}{Definicja}[section]
\newtheorem*{ndefi}{Definicja}
\newtheorem{theorem}{Twierdzenie}[section]

\newcommand{\V}[1]{\ensuremath{\mathbf{#1}}}
\newcommand{\T}[1]{\ensuremath{ {#1}^{\mathbf{{T}}}}}
\newcommand{\Tnawiasy}[1]{\ensuremath{ ({#1})^{\mathbf{{T}}}}}
\newcommand{\M}[1]{\ensuremath{\mathbf{#1}}}
\newcommand{\Sp}[1]{\ensuremath{\mathcal{#1}}}
\newcommand{\SpG}[1]{\ensuremath{\langle {#1} \rangle }}
\newcommand{\cialo}[1]{\ensuremath{\mathcal{#1}}}

\newcommand{\nxn}{\ensuremath{ n \times n }}

\newcommand{\prodA}[2]{\ensuremath{ \langle #1 , #2 \rangle_{\M{A}}}}
\newcommand{\prodAA}[2]{\ensuremath{ \T{#1} \M{A} #2 }}
\newcommand{\prodAATnawiasy}[2]{\ensuremath{ \Tnawiasy{#1} \M{A} #2 }}
\newcommand{\dsum}{\displaystyle\sum}

\newcommand{\Axmb}{ \M{A} \V{x} - \V{b}}
\newcommand{\Adxmb}{ \M{A} \V{\dot{x}} - \V{b}}
\newcommand{\dx}{\ensuremath{\dot{\V{x}}}}

%rzutowanie (projekcja)
\newcommand{\projA}[2]{\ensuremath{ {proj_{#1} (#2)   }}}
\newcommand{\projAWzor}[2]{\ensuremath{ {\displaystyle\frac{\prodA{#1}{#2}}{ \prodA{#1}{#1}}  #1 }}}
\newcommand{\projAWzorM}[2]{\ensuremath{#1 ( \prodAA{#1}{#1} )^{-1} \T{#1} \M{A} #2 }}
\newcommand{\projAWzorMTnawiasy}[2]{\ensuremath{#1 ( \prodAATnawiasy{#1}{#1} )^{-1} \Tnawiasy{#1} \M{A} #2 }}

\newcommand{\Winv}[1]{\ensuremath{ \M{W}_{#1}^{\mathtt{inv}} }}

\newcommand{\rank}[1]{\ensuremath{ rank(#1)}}
\newcommand{\Zdwa}{\ensuremath{ \mathbb{Z}_2}}
\newcommand{\Zdwan}{\ensuremath{ {\displaystyle\mathbb{Z}_2^n}}}

\newcommand{\AWSTV}[1]{\ensuremath{\M{A}\M{W}_{#1} \T{\M{S}_{#1}} + \M{V}_{#1} }}
\newcommand{\VWinvVT}[1]{\ensuremath{\M{V}_{#1} \Winv{#1} \T{\M{V}_{#1}}  }}
\newcommand{\VTAVWinv}[1]{\ensuremath{\T{\M{V}_{#1}} \M{A} \M{V}_{#1} \Winv{#1}   }}
\newcommand{\VSST}[1]{\ensuremath{\M{V}_{#1} \M{S}_{#1} \T{\M{S}_{#1}}  }}
\newcommand{\VS}[2]{\ensuremath{\M{V}_{#1} \M{S}_{#2} }}
\newcommand{\STVT}[2]{\ensuremath{\T{\M{S}_{#1}} \T{\M{V}_{#2}} }}
\newcommand{\OWpppW}[2]{\ensuremath{ \Sp{O}( \Sp{W}_{#1} + ... +  \Sp{W}_{#2}) }}
\newcommand{\OWW}[2]{\ensuremath{ \Sp{O}( \Sp{W}_{#1}  +  \Sp{W}_{#2}) }}

%\newcommand{\myurl}[1]{\textit{#1}}
\newcommand{\myurl}[1]{\\ \url{#1}}

\newcommand{\rysunek}[3] {
\begin{figure}[h]
    \centering
    \includegraphics[width=0.8\textwidth]{#1}
    \caption{#2}
    #3
\end{figure}
}

\newcommand{\rysunekHT}[3] {
\begin{figure}[ht]
    \centering
    \includegraphics[width=0.8\textwidth]{#1}
    \caption{#2}
    #3
\end{figure}
}

\newcommand{\rysunekW}[4] {
\begin{figure}[h]
    \centering
    \includegraphics[width=#4\textwidth]{#1}
    \caption{#2}
    #3
\end{figure}
}

% koniec definicji

\begin{document}
\maketitle

\begin{abstract}
W pracy przedstawiono prototyp urządzenia potrafiącego określić swoje
położenie jak i orientację w przestrzeni. Urządzenie składa się z dwóch części:
nadajnika, którego pozycja i orientacja jest wyznaczana oraz odbiornika, który 
stanowi stały punkt odniesienia. Wyznaczanie położenia odbywa się za 
pomocą pomiarów odległości pomiędzy nadajnikiem, a odbiornikiem wykorzystując 
do tego celu ultradźwięki o częstotliwości \SI{40}{kHz}. 
Zastosowana metoda umożliwia pomiar z wysoką rozdzielczością dochodzącą do \SI{0.5}{mm}.
\end{abstract}


\tableofcontents
%\listoffigures
%\listoftables


\chapter*{Wprowadzenie}
\addcontentsline{toc}{chapter}{Wprowadzenie}

W dzisiejszym świecie wyraźnie wzrasta zapotrzebowanie na urządzenia,
które potrafią określić swoje położenie i orientację w przestrzeni.
Urządzenia takie mają szerokie zastosowanie w~wielu dziedzinach, m.in. 
w~wirtualnej rzeczywistości, rozszerzonej rzeczywistości, kartografii lub 
 podczas trójwymiarowego skanowania obiektu.
Przykładowo okulary do wirtualnej rzeczywistości, takie jak \textit{Oculus Rift} \cite{bib:OculusRift} 
czy \textit{castAR} \cite{bib:castAR},
muszą uwzględnić położenie i orientację głowy, aby na tej podstawie wyświetlić użytkownikowi 
odpowiednią treść. 


Powstało wiele rozwiązań tego problemu. Do najczęściej stosowanych możemy zaliczyć:
\begin{enumerate}
 \item 
 Wykorzystanie akcelerometrów, żyroskopów i magnetometrów -- 
takie rozwiązanie zastosowano w \textit{Oculus Rift development kit}. Zaletami tej metody są stosunkowo
prosta konstrukcja i niska cena.
Do wad należy zaliczyć brak stałych punktów odniesienia, co skutkuje występowaniem tzw. dryftu.
\textit{Oculus Rift development kit} radzi sobie z tym problemem, modelując w komputerze możliwe zmiany pozycji głowy.
Jednak rozwiązanie to jest dalekie od idealnego, o czym może świadczyć fakt, że w kolejnej wersji 
urządzenia dodano zewnętrzną kamerę śledzącą pozycję głowy.

\item \label{itm:second_method}
 Projektowanie światła strukturalnego na otoczenie i zbieranie informacji o strukturze 
 światła odbitego za pomocą sensorów, zazwyczaj kamer -- 
 taką metodę wykorzystano w \textit{Microsoft Kinnect} \cite{bib:MicrosoftKinect}.
 Urządzenie wyświetla na otoczenie stały wzór punktów, następnie kamerą na podczerwień
 zbiera informację o zniekształceniu danego wzoru i na tej podstawie odtwarza  
 trójwymiarową strukturę otoczenia i~położenie urządzenia w~tym otoczeniu.
 Podobną metodę wykorzystuje \textit{Oculus Rift development kit 2} \cite{bib:OculusRiftDK2} oraz 
 \textit{castAR} \cite{bib:castAR} -- tutaj za źródła światła służą diody podczerwone umieszczone na okularach.
 Emitowane przez nie światło rejestruje kamera umieszczona przed użytkownikiem.
 Komputer na podstawie względnego położenia widocznych punktów określa położenie i~orientację
 okularów w przestrzeni.
 Zaletą tego rozwiązania są stałe punkty odniesienia (kamera), a także możliwość pomiaru wielu punktów naraz.
 Wadami są stosunkowo niska rozdzielczość, szczególnie w osi Z, i duży strumień danych do obróbki.

\item
 Wykorzystanie wielu zdjęć, na postawie których 
 wyznaczana jest pozycja kamery względem znajdujących się na nich  stałych (niezmieniających się w czasie) obiektów. 
 Możliwa jest też odwrotna sytuacja, gdy  
  kamera (lub wiele kamer) jest punktem stałym i względem niej wyznaczana jest pozycja fotografowanych obiektów --   
 taką metodę wykorzystano w \textit{VidialSFM} \cite{bib:VisualSFM} oraz w \textit{The Pi 3D scanner project} \cite{bib:pi3dscan}. 
 Wadą tego rozwiązania jest dość niska rozdzielczość.
 
 \item
 Pomiar czasu, jakiego potrzebują fale elektromagnetyczne na dotarcie od nadajników o znanych położeniach 
 do obiektu, którego położenie nas interesuje.
 Dzięki znajomości czasów oraz prędkości rozchodzenia się fal można wyznaczyć odległości pomiędzy nadajnikami a obiektem.
 Na podstawie odległości wyznaczana jest lokalizacja przestrzenna obiektu. 
 Tę metodę powszechnie stosuje się takich systemach jak GPS \cite{bib:gps}, GLONASS \cite{bib:GLONASS} czy Galileo \cite{bib:galileo}.
 Jej zaletą jest możliwość lokalizowania obiektów na dużych przestrzeniach.
 Jednak ze względu na dużą prędkość, z jaką rozchodzą się fale elektromagnetyczne, metoda ta cechuje się dość niską 
 dokładnością nieprzekraczającą \SI{10}{cm}.
 
 \item
 Pomiar siły sygnału radiowego (fal elektromagnetycznych) docierającego do obiektu, którego lokalizacja nas interesuje.
 Źródłem sygnału mogą być dedykowane radiolatarnie, np. \textit{Estimote Beacon} \cite{bib:beacon},
 czy inne nadajniki, np. WiFi \cite{bib:lokWiFi}.
 Wadę tego rozwiązania stanowi niska dokładność pomiaru, która wynosi maksymalnie kilkadziesiąt centymetrów.  
 
 \item
 Wykorzystanie światła podczerwonego naturalnie emitowanego przez badany obiekt --
 takie podejście zaprezentowano w pracy \textit{PIR Sensors: Characterization and Novel Localization Technique}
 \cite{bib:PIRsens}. Ta metoda jest podobna do metody drugiej, z tą różnicą, że zrezygnowano w niej
 ze sztucznego źródła światła, zamiast niego wykorzystując naturalne światło podczerwone emitowane 
 przez badany obiekt. Tutaj również uzyskuje się stosunkowo niską dokładność pomiarową.
 
 \item
 Wykorzystanie indukcji magnetycznej --  metoda polega na pomiarze siły zmiennego pola magnetycznego niskiej częstotliwości
 generowanego przez zewnętrzne anteny. Na podstawie siły pola docierającego do badanego obiektu estymowane są jego odległości
 od anten, a na ich podstawie wyznaczana jest lokalizacja przestrzenna obiektu.
 Główną zaletę tego podejścia stanowi fakt, że pole magnetyczne przenika przez większość materiałów. Pozwala  
 to na lokalizowanie obiektu również pod ziemią, co zaprezentowano w pracy
 \textit{Revealing the hidden lives of underground animals using magneto-inductive tracking} \cite{bib:chomiki}. 
 Wadą metody jest mały zasięg, ponieważ pole magnetyczne propaguje się odwrotnie proporcjonalnie do sześcianu  
 odległości od źródła.
 
 \item 
 Wykorzystanie specyficznych warunków zewnętrznych, w jakich może znajdować się obiekt, którego lokalizacja nas interesuje
 -- w pracy \textit{We Can Track You If You Take the Metro: Tracking Metro
Riders Using Accelerometers on Smartphones} \cite{bib:metro} autorzy opisali algorytm lokalizowania
użytkownika metra na podstawie danych z akcelerometru, które zmieniały się podczas jazdy.
Metoda ta jest jednak dość specyficzna i nie nadaje się do ogólnych zastosowań. 
 
\end{enumerate}
 W niniejszej pracy przedstawiono prototyp, który jest oparty na zmodyfikowanej czwartej metodzie i zamiast 
 fal elektromagnetycznych wykorzystuje ultradźwięki. 
 Podejście to zapewnia dużą dokładność, szczególnie w osi Z, prostotę budowy, a także niską cenę.
 Urządzenie składa się z dwóch części: odbiornika, na którym umieszczono trzy mikrofony, oraz nadajnika,
 na którym znajdują się cztery głośniki. Pomiędzy mikrofonami i głośnikami mierzy się odległość z rozdzielczością
 dochodzącą do \SI{0,5}{mm}, dzięki czemu prototyp jest w stanie określić położenie nadajnika
w przestrzeni i jego orientację. 
Pomiaru odległości dokonuje się za pomocą ultradźwięków o częstotliwości \SI{40}{kHz}.
Mimo iż ta metoda jest znana od wielu lat, to rzadko się ją stosuje do precyzyjnych pomiarów
z uwagi na relatywnie dużą długość fali ultradźwiękowej,
która dla częstotliwości \SI{40}{kHz} wynosi około \SI{8.5}{mm}.

Autorowi niniejszej pracy udało się to pozorne ograniczenie przezwyciężyć.
Ostatecznie urządzenie jest w stanie śledzić położenie nadajnika z rozdzielczością 
\SI{5000}{px} $\times$ \SI{5000}{px} $\times$ \SI{5000}{px} w przestrzeni ograniczonej sześcianem o rozmiarach 
\SI{2,5}{m} $\times$ \SI{2,5}{m}  $\times$ \SI{2,5}{m}, przy czasie odświeżania rzędu \SI{350}{ms}.
Warto podkreślić, że przyjęte parametry sześcianu wynikają ze względów praktycznych i bez 
problemu można je zwiększyć, zachowując stosunek rozdzielczości na metr.
Relatywnie długi czas odświeżania zależy głównie od czasu,  
 jakiego potrzebuje fala dźwiękowa, by się rozproszyć,
 tak by jej odbicia od powierzchni ścian nie wpływały na kolejne pomiary.

Należy również wspomnieć o innych pracach, które w podobny sposób podchodzą do problemu pozycjonowania. Przykładowo 
w pracy \textit{Ultrasonic 3D Wireless Computer Mouse} \cite{bib:mouse} przedstawiono prototyp myszki 3D, której
pozycja w przestrzeni określana jest za pomocą ultradźwięków, niemniej wykorzystany przez autorów algorytm wyznaczania 
odległości jest dużo bardziej podatny na błędy od metody prezentowanej w niniejszej pracy. Ponadto autorzy koncentrują się jedynie
na określaniu położenia myszki w przestrzeni, bez wyznaczania jej orientacji.
Kolejne ciekawe rozwiązanie stanowi rękawica dla graczy 
\textit{Power Glove} \cite{bib:powerGlove}, która pojawiła się w sprzedaży w 1989 roku \cite{bib:powerGlove2}. 
Ona również wykorzystuje ultradźwięki do wyznaczania pozycji
w przestrzeni, jednak nie odniosła  znaczącego sukcesu komercyjnego ze względu na wyjątkowo niską precyzję urządzenia.



\chapter{Podstawy teoretyczne}
\section{Wyznaczanie położenia na podstawie odległości od punktów stałych}

Zadaniem prezentowanego prototypu jest wyznaczenie położenia głowicy w przestrzeni na podstawie
podległości od znanych punktów stałych, niech $\boldsymbol{x}$ będzie szukanym punktem w przestrzeni,
$\boldsymbol{y_1,y_2,y_3}$ stałymi punktami o znanym położeniu w przestrzeni, a $r_1,r_2,r_3$ wyznaczonymi odległościami, 
rysunek \ref{fig:polozenie}. Szukane położenie punktu $\boldsymbol{x}$ możemy wyznaczyć za pomocą układu równań:
\[
 \begin{cases}
    |\boldsymbol{y_1} - \boldsymbol{x}| = r_1
 \\ |\boldsymbol{y_2} - \boldsymbol{x}| = r_2
 \\ |\boldsymbol{y_3} - \boldsymbol{x}| = r_3
 \end{cases}
\]
Zauważmy, że znając odległość $\boldsymbol{x}$ od jednego punktu stałego wiemy, że $\boldsymbol{x}$ będzie leżał na
sferze o promieniu równym danej odległości. Znając dwie odległości możemy ograniczyć poszukiwania do części wspólnej dwóch sfer,
czyli okręgu. Dla znanych trzech odległości dostajemy cześć wspólną okręgu ze sferą, czyli dwa możliwe punkty.
Mogłoby się wydawać, że tyle informacji nie jest wystarczających do jednoznacznego określenia położenia
szukanego punktu, jednak te dwa punktu są na ogół od siebie daleko oddalone i bez większych problemów
możemy założyć, że jedno z rozwiązań jest naszym szukanym punktem.

\rysunekW{polozenie}{wyznaczenie położenia na podstawie odległości od punktów stałych}{\label{fig:polozenie}}{0.7}

Powyższy układ równań rozwija się do:
\[
 \begin{cases}
     (y_{11}-x_1)^2 + (y_{12}-x_2)^2 + (y_{13}-x_3)^2 = r_1^2
 \\  (y_{21}-x_1)^2 + (y_{22}-x_2)^2 + (y_{23}-x_3)^2 = r_2^2
 \\  (y_{31}-x_1)^2 + (y_{32}-x_2)^2 + (y_{33}-x_3)^2 = r_3^2
 \end{cases}
\]

Ponieważ mamy pewną powolność w doborze punktów $\boldsymbol{y_i}$ dla uproszenia możemy przyjąć, że:
$\boldsymbol{y_1}=(\frac{a}{2},0,0), \boldsymbol{y_2}=(-\frac{a}{2},0,0), \boldsymbol{y_3}=(0,h,0)$, czyli są wierzchołkami
trójkąta równoramiennego o wysokości $h$ i podstawie $a$, wtedy układ równań upraszcza
się do postaci:
\[
 \begin{cases}
     (\frac{a}{2}-x_1)^2 + x_2^2 + x_3^2 = r_1^2
 \\  (\frac{a}{2}+x_1)^2 + x_2^2 + x_3^2 = r_2^2
 \\  x_1^2 + (h-x_2)^2 + x_3^2 = r_3^2
 \end{cases}
\]
z czego ostatecznie otrzymujemy:
\[
 \begin{cases}
     x_1 = \frac{r_2^2 - r_1^2}{2a}
 \\  x_2 = \frac{6r_2^2 - 4r_3^2 - 2r_1^2 + a^2}{8h}  + \frac{h}{2}
 \\  x_3 = \pm \sqrt{r_3^2-x_1^2-(h-x_2)^2}
 \end{cases}
\]






\section{Pomiar odległości za pomocą ultradźwięków}

Prezentowany prototyp do pomiaru odległości wykorzystuje metodę pomiaru czasu jaki 
potrzebuje sygnał ultradźwiękowy aby pokonać drogę od nadajnika do odbiornika,
rysunek \ref{fig:pomiar_odleglosci}.
\rysunekW{pomiar_odleglosci}{pomiar odległości za pomocą ultradźwięków}{\label{fig:pomiar_odleglosci}}{0.4}

Znając prędkość rozchodzenia się fal dźwiękowych w powietrzy oraz czas jaki fala dźwiękowa potrzebowała
aby pokonać dany dystans jesteśmy w stanie wyznaczyć szukaną odległość.

Prędkość dźwięku w powietrzu jest mocno zależy od panujących warunków atmosferycznych \cite{bib:soundSpeed},  
głównym czynnikiem na nią wpływającym jest temperatura.
dla gazu doskonałego prędkość wyraża się wzorem:
\[
V_{air} = 331.3  \frac{m}{s}  \sqrt{1+\frac{T}{273.15}}
\]
gdzie: $T$ jest temperaturą w stopniach Celsjusza ($^\circ$C).
Wzór ten możemy przybliżyć za pomocą rozwinięcia Taylora dla temperatur bliskich temperaturze pokojowej ($25^\circ$C):
\[
 V_{air} = (346.13  +  0.580(T - 25^\circ)) \frac{m}{s}
\]

Mimo iż współczynnik temperaturowy jest stosunkowo mały i stanowi jedynie 0.17\% całej prędkości
to przy pomiarach odległości rzędu metrów i rozdzielczościach rzędu milimetrów staje się istotny. 
Dlatego w prezentowanym prototypie istotną częścią stanowi kalibracja wstępna, podczas której
wyznaczana jest aktualna prędkość dźwięku, ubocznym efektem kalibracji jest pomiar temperatury otoczenia.
Przez zależność zależność temperaturową urządzenie nadaje się jedynie do zastosowań w pomieszczeniach
w których nie zmienia się temperatura otoczenia w czasie pomiarów.






\chapter{Budowa prototypu}

Prezentowany prototyp składa się z dwóch części: nadajnika i odbiornika, pomiędzy którymi
dokonywany jest pomiar odległości. W niniejszym rozdziale przedstawiono zasadę funkcjonowania 
poszczególnych komponentów.

\section{Budowa i zasada działania nadajnika}

Nadajnik zbudowany został na bazie płytki prototypowej \textit{Arduino Nano} \cite{bib:arduinoNano},
która składa się z procesora ATmega328 \cite{bib:atmega328} taktowanego rezonatorem kwarcowym \SI{16}{MHz},
stabilizatora napięcia \SI{5}{V} oraz układu FT232RL umożliwiającego 
jej programowanie  ze środowiska \textit{Arduino} \cite{bib:Arduino} za pośrednictwem portu USB. 
\textit{Arduino Nano} połączono 
bezpośrednio z czterema nadajnikami ultradźwiękowymi (głośnikami, rezonatorami piezoelektrycznymi) typu 40ST-12 \cite{bib:40ST12}.
Za pośrednictwem złącza SV1 do płytki doprowadzono zasilanie oraz sygnały sterujące z odbiornika. 
Schemat połączeń przedstawia rysunek \ref{fig:nadajnik_schemat}.

 \begin{figure}[h]
    \centering
    \includegraphics[width=0.6\textwidth, trim= 0mm 0mm 0mm 0mm,clip]{transmitter}
    \caption{Schemat nadajnika}
    \label{fig:nadajnik_schemat}
\end{figure}

Całość umieszczono na ramie w kształcie litery H wykonanej z rurek PCV (rysunek \ref{fig:nadajnik_szkic}).
Rezonatory dodatkowo odizolowano  od ramy przy użyciu rzepów, co ułatwia ich demontaż, a także skutecznie
zapobiega przenoszeniu się drgań. 

\rysunek{nadajnik_H}{Szkic ramy nadajnika}{\label{fig:nadajnik_szkic}}


\textit{Android Nano} połączony jest z odbiornikiem sześciometrowym kablem, którym przesyłane są sygnały sterujące oraz zasilanie.
Do sterowania wykorzystano trzy przewody -- dwa z nich informują, który z głośników ma w danym momencie nadawać,
a trzeci służy jako wyzwalacz. 
Informacja, który z głośników  ma nadawać, kodowana jest za pomocą dwóch bitów w systemie binarnym --
napięcie na przewodzie \SI{3,3}{V} oznacza logiczną jedynkę, a \SI{0}{V} oznacza logiczne zero.

Na potrzeby nadajnika powstało oprogramowanie w C dla procesora ATmega328 generujące nadawany sygnał.
Cała logika programu mieści się w obsłudze przerwania sprzętowego, które reaguje na opadające zbocze na wyzwalaczu.
Gdy następuje przerwanie, oprogramowanie wysyła z góry zdefiniowany sygnał do odpowiedniego rezonatora. 
Nadawany sygnał został tak dobrany, by dało się go w prosty sposób wyodrębnić oraz by trwał jak najkrócej. Składa się z dwóch
części: wzbudzającej i tłumiącej.
Okres impulsów sygnału jest zgodny z częstotliwością rezonansową przetworników, a dodatkowo część tłumiąca
jest przesunięta względem części wyzwalającej o 180 stopni.
Czas trwania części wzbudzającej jest najkrótszy z możliwych i trwa jeden okres, natomiast czas trwania części tłumiącej określono
tak, by odebrany sygnał zawierał również sinusoidy przesunięte fazowo względem początkowej części sygnału.
Rysunek \ref{fig:output_signal} przedstawia sygnał podawany na przetwornik piezoelektryczny.

\begin{figure}[ht]
    \centering
    \includegraphics[width=1.0\textwidth, trim= 25mm 0mm 0mm 0mm,clip]{output_signal}
    \caption{Sygnał podawany na przetwornik piezoelektryczny}
    \label{fig:output_signal}
\end{figure}

\section{Dobór rezonatorów piezoelektrycznych}

Głównym problemem podczas konstrukcji nadajnika okazał się dobór odpowiednich rezonatorów piezoelektrycznych.
Mimo że producent wykorzystanych rezonatorów zapewnia ich pracę w zakresie \SI{40}{kHz} $\pm$ \SI{1}{kHz},
to taki rozrzut okazał się zbyt duży, 
dlatego z 30 rezonatorów (15 nadajników i 15 odbiorników) wybrane zostały 4 nadajniki i 3 odbiorniki o najbardziej 
zbliżonych częstotliwościach rezonansowych.
Tabela \ref{table:czestotliwosci} zawiera wyniki pomiarów częstotliwości. Gwiazdką oznaczono wykorzystane przetworniki piezoelektryczne.

\begin{table}[h]
  \caption{Częstotliwości rezonansowe przetworników piezoelektrycznych}
  \smallskip
  \label{table:czestotliwosci}
  \centering
  \begin{tabular}{|r|r|r|}
    \hline 
    Nr & Nadajnik 40ST-12 & Odbiornik 40SR-12\\
    \hline
    1  &   \SI{40,88}{kHz} & *\SI{40,65}{kHz} \\
    2  &   \SI{41,12}{kHz} &  \SI{40,45}{kHz} \\
    3  &  *\SI{40,78}{kHz} &  \SI{39,52}{kHz} \\
    4  &   \SI{41,19}{kHz} &  \SI{40,47}{kHz} \\
    5  &   \SI{40,92}{kHz} &  \SI{40,66}{kHz} \\
    6  &   \SI{39,68}{kHz} & *\SI{40,69}{kHz} \\
    7  &   \SI{39,78}{kHz} &  \SI{40,59}{kHz} \\
    8  &  *\SI{40,80}{kHz} &  \SI{40,39}{kHz} \\
    9  &   \SI{40,90}{kHz} &  \SI{40,29}{kHz} \\
    10 &  *\SI{40,66}{kHz} & *\SI{40,68}{kHz} \\
    11 &  *\SI{40,85}{kHz} &  \SI{39,22}{kHz} \\
    12 &   \SI{41,01}{kHz} &  \SI{39,51}{kHz} \\
    13 &   \SI{41,00}{kHz} &  \SI{39,92}{kHz} \\
    14 &   \SI{39,82}{kHz} &  \SI{39,26}{kHz} \\
    15 &   \SI{39,64}{kHz} &  \SI{39,11}{kHz} \\
    \hline
  \end{tabular}
\end{table}



\chapter{Odbiornik}


Centralną częścią prezentowanego prototypu jest odbiornik,
jego zadaniem jest wysyłanie sygnałów sterujących do nadajnika, zbieranie ultradźwięków z otoczenia oraz przesłanie
ich do dalszej analizy do komputera.
Odbiornik składa się z sześciu części (rysunek \ref{fig:odbiornik_szkic}):

\begin{itemize}
 \item trzech modułów ultradźwiękowych przetwarzających ultradźwięki na sygnał elektryczny
 \item płytki prototypowej \textit{stm32f4-discovery} \cite{bib:stm32f4Discovery}, która odpowiada za komunikację z komputerem
 \item przystawki do \textit{stm32f4-discovery} przystosowującej sygnały elektryczne z modułów ultradźwiękowych
  do poziomów akceptowalnych przez \textit{stm32f4-discovery}
 \item ramy, na której umieszczone są moduły ultradźwiękowe
\end{itemize}


\rysunek{odbiornik_szkic}{szkic odbiornika}{\label{fig:odbiornik_szkic}}


\section{Budowa i zasada działania}

Głównym elementem odbiornika stanowi płytka prototypowa \textit{stm32f4-discovery} \cite{bib:stm32f4Discovery},
jej zadaniem jest odbieranie ultradźwięków z otoczenia i przesłanie ich do dalszej analizy do komputera oraz wysyłanie sygnałów wyzwalających
do nadajnika.

Oprogramowanie płytki prototypowej oparte jest na bibliotece \textit{stm32 usb 101} \cite{bib:stm32_usb_101}
zapewniającej komunikację z komputerem, do której dodana została obsługa przetworników ADC.
Program poprzez port USB dostaje komunikat, który z czterech nadajników ma nadawać, ta informacja przekazywana jest
dalej do nadajnika w raz z sygnałem wyzwalającym, następnie uruchamiane są trzy przetworniki ADC, które 
samplują odbierany dźwięk i poprzez DMA zapisują trzy kanały w pamięci procesora.
Częstotliwość pracy przetworników ustawiona jest na 1.6MSPS co daje średnio 40 sampli na jedną sinusoidę nadanego 40kHz sygnału.
Program zapamiętuje 16kS na każdym kanale, co przy prędkości dźwięku 340m/s daje maksymalną mierzoną odległość rzędu 3.5 metrów.
Po zebraniu w sumie 48kS, całość przesyłana jest do komputera w celu dalszej analizy.
Cały proces powtarzany jest dla każdego z czterech przetworników piezoelektrycznych, 
co w sumie daje 12 sygnałów na podstawie których wyznaczona zostaje 
pozycja w przestrzeni oraz orientacja nadajnika.

Cała elektronika osadzona została na ramie zbudowanej z rur PCV w kształcie trójkąta (rysunek \ref{fig:trojkat}), 
Odległości pomiędzy odbiornikami są z góry znane, co ułatwia obliczenia.

\rysunek{trojkat}{szkic ramy odbiornika}{\label{fig:trojkat}}



\clearpage
\section{Budowa odbiornika ultradźwiękowego}

Fale dźwiękowe przetwarzane są na sygnał elektryczny za pomocą odbiornika ultradźwiękowego (piezoelektrycznego) 
podłączonego do dwóch wzmacniaczy operacyjnych. Schemat przedstawiony jest na rysunku \ref{fig:odbiornik_ultra}.

\rysunek{receiver}{schemat odbiornika ultradźwiękowego}{\label{fig:odbiornik_ultra}}

Pierwszy ze wzmacniaczy pracuje jako przedwzmacniacz ładunkowy \cite{bib:wzm_ladunkowy}.
Ładunek wytworzony na przetworniku piezoelektrycznym zostaje w całości przeniesiony na kondensator $C2$ 
(wzmacniacz stara się utrzymać różnicę potencjałów miedzy dodatnim a ujemnym wejściem na zerowym poziomie),
co jest równoważne z pojawieniem się napięcia na kondensatorze zgodnie z równaniem $U=\frac{q}{C}$.
Rezystor $R1$ ustawia napięcia spoczynkowe układu na poziome $\frac{1}{2}$ Vcc jak i rozładowuje kondensator $C2$.
$R1$ i $C2$ działają również jako filtr górnoprzepustowy.

Drugi wzmacniacz pracuje jako zwykły wzmacniacz napięciowy wzbogacony o filtr górno i dolnoprzepustowy.

Aby zminimalizować zakłócenia całość oparta jest na niskoszumowym wzmacniaczu operacyjnym NE5532 \cite{bib:ne5532}, 
dodatkowo cała płytka $PCB$ jest ekranowana.

Wzmocniony sygnał poprzez wtyczkę $JP1$ doprowadzony jest do \textit{shieldu} współpracującego z \textit{stm32f4-discovery}.

\clearpage

\section{\textit{Shield} do \textit{stm32f4-discovery}}

Sygnał z odbiorników doprowadzany jest do \textit{stm32f4-discovery} za pomocą specjalnej przystawki (\textit{shieldu}).
Jej zadaniem jest przystosowanie amplitud zebranych sygnałów do wartości akceptowalnych przez STM32F407VGT6,
dodatkowo przystawka zawiera układ zasilający jak i wyprowadzone są z niej sygnały sterujące do nadajnika.
Zakres napięć dopuszczalnych dla przetowrnika analogowo-cyfrowego zawierają się w przedziale od 0V do 3.3V,
dlatego zastosowany został układ TLV2774 \cite{bib:TLV2774}, który zawiera w sobie 4 wzmacniacze operacyjne typu
rail-to-rail. Wzmacniacze pracują w układzie odwracającym, do którego dodano regulowane napięcie odniesienia.
Schemat przystawki przedstawiony jest na rysunku \ref{fig:shild}.



\rysunek{mainboard2}{schemat \textit{shieldu} do \textit{stm32f4-discovery}}{\label{fig:shild}}



\chapter{Przetwarzanie danych zebranych z~odbiornika}

W ramach niniejszej pracy powstało oprogramowanie w \textit{Pythonie}, którego zadaniem jest
przekształcenie surowych danych ultradźwiękowych 
z odbiornika na informację o położeniu nadajnika w przestrzeni oraz jego orientacji.
Na oprogramowanie składają się biblioteka \textit{mp3d} zajmująca się
sterowaniem odbiornikiem, odbieraniem sygnałów ultradźwiękowych i ich analizą, program \textit{scan.py}
wizualizujący dane w postaci obrazu trójwymiarowego oraz program \textit{save-pattern.py},
który służy do wstępnej kalibracji urządzenia.

Najistotniejszą częścią oprogramowania jest biblioteka \textit{mp3d} podzielona na pięć modułów:
\begin{itemize}
 \item \textit{com.py} --  odpowiedzialny za komunikację z odbiornikiem,
 \item \textit{find\_pattern.py} --  wyznaczający odległości przez wyszukanie wzorca w odebranym sygnale,
 \item \textit{xyz.py} --  określający pozycję i orientację nadajnika oraz  weryfikujący, 
 czy zebrane dane odpowiadają modelowanej rzeczywistości,
 \item \textit{info.py} --  wyświetlający informację o sile odbieranego sygnału,
 \item \textit{ply.py} --  odpowiedzialny za eksportowanie danych do formatu \textit{.ply},
    (Polygon File Format) obsługiwanego przez większość programów do obróbki grafiki 3D.
\end{itemize}


\section{Komunikacja z odbiornikiem, moduł \textit{com.py}}

Za komunikację z odbiornikiem odpowiedzialny jest moduł \textit{com.py}.
Pracuje on w oddzielnym wątku, w którym cyklicznie wysyłane są żądania, by dany głośnik nadał sygnał,
a następnie odbierane są sygnały z trzech mikrofonów.
Ten proces odbywa się kolejno dla każdego z~czterech głośników nadajnika.
Zebrane w ten sposób dwanaście sygnałów po wstępnej filtracji przekazywane są dalej do \textit{find\_pattern.py}. Cały cykl powtarza 
się co \SI{350}{ms}. Rysunek \ref{fig:com_output_2m} przedstawia sygnał odebrany z jednego z mikrofonów.


\begin{figure}[h!]
    \centering
    \includegraphics[width=1.13\textwidth, trim= 53mm 0mm 0mm 0mm,clip]{com_output_2m_1}
    \includegraphics[width=1.13\textwidth, trim= 53mm 0mm 0mm 0mm,clip]{com_output_2m_2}
    \caption{Sygnał odebrany przez moduł \textit{com.py}. 
    Odległość między nadajnikiem a odbiornikiem wynosi 2 metry.
    Na górnym wykresie po prawej stronie widać również sygnały odbite od ścian}
    \label{fig:com_output_2m}
\end{figure}

\newpage
\section{Wyznaczanie odległości, moduł \textit{find\_pattern.py}}

Moduł \textit{find\_pattern.py} odpowiada za obliczenie odległości pomiędzy czterema głośnikami i~trzema mikrofonami.
Wyznaczana jest ona przez wyszukanie w każdym z 12 odebranych sygnałów  położenia \textit{wzorca}, który 
odpowiada czołu nadanego sygnału (rysunek \ref{fig:com_output_2m}).
\textit{Wzorzec} po raz pierwszy wprowadza się podczas kalibracji, a następnie ciągle aktualizuje 
(podmienia na sygnałem pasującym do \textit{wzorca}).

W celu wyszukania \textit{wzorca} wewnątrz sygnału korzysta się z  metody najmniejszego błędu średniokwadratowego:
Niech $w(t)$  dla $t = 0, ..., n-1$ będzie szukanym \textit{wzorcem}, a $f(x)$ dla $x = 0, ..., m-1$ odebranym sygnałem.
Wtedy możemy znaleźć takie $a$, że błąd średniokwadratowy $E(x)$ pomiędzy $w(t)$ i $a f(t+x)$ jest minimalny.
Możemy to zapisać w postaci
\[
  E(x) = \min_{a \in R} \{ \sum_{t=0}^{n-1}  (w(t) - a f(t+x))^2 \}.
\]
Po podniesieniu wyrażenia w nawiasie do kwadratu otrzymujemy
\[
  E(x) = \min_{a \in R} \{ \sum_{t=0}^{n-1}  (w^2(t) -2a w(t) f(t+x) + a^2 f^2(t+x)) \},
\]
\[
  E(x) = \min_{a \in R} \{ \sum_{t=0}^{n-1}  w^2(t) -2a \sum_{t=0}^{n-1}  w(t) f(t+x) + a^2 \sum_{t=0}^{n-1} f^2(t+x) \}.
\]
Aby wyliczyć $E(x)$, należy zminimalizować wyrażenie
\[
  y(a) = \sum_{t=0}^{n-1}  w^2(t) -2a \sum_{t=0}^{n-1}  w(t) f(t+x) + a^2 \sum_{t=0}^{n-1} f^2(t+x).
\]
Korzystając z faktu, że $y(a)$ jest funkcją kwadratową, możemy wyliczyć $a$ minimalizujące $y(a)$
\[
  a = \frac{ \sum\limits_{t=0}^{n-1}  w(t) f(t+x) }{ \sum\limits_{t=0}^{n-1} f^2(t+x) }.
\]
Ostatecznie dostajemy wzór na $E(x)$
\[
  E(x) = \sum_{t=0}^{n-1}  w^2(t)  - \frac {\left(\sum\limits_{t=0}^{n-1}  w(t) f(t+x) \right)^2 } { \sum\limits_{t=0}^{n-1} f^2(t+x)}.
\]

Zauważmy, że dzięki skalowaniu skalarem $a$  sygnału $f(x)$, a nie \textit{wzorca} $w(x)$,
otrzymane $E(x)$ nie zależy od siły odebranego sygnału. Ułatwia to porównanie błędów w dwóch różnych miejscach.
$E(x)$ zależy jednak od siły sygnału wzorcowego -- możemy się od niego  uniezależnić, wyznaczając
błąd względny $e(x)$
\[
  e(x) = \frac{E(x)}{\sum\limits_{t=0}^{n-1}  w^2(t)}.
\]
Po podstawieniu $E(x)$ dostajemy
\[
  e(x) = 1 - \frac {\left(\sum\limits_{t=0}^{n-1}  w(t) f(t+x) \right)^2 } { \sum\limits_{t=0}^{n-1} f^2(t+x) \sum\limits_{t=0}^{n-1}  w^2(t)}.
\]
 
 Funkcję $e(x)$  można interpretować następująco:
 im mniejszy błąd $e(x)$, tym większe prawdopodobieństwo, że szukany \textit{wzorzec} $w$ znajduje się na pozycji $x$ w 
 odebranym sygnale $f$. 

 Wynikiem obliczeń modułu \textit{find\_pattern.py} jest cała funkcja $e(x)$, na podstawie której moduł \textit{xyz.py}
 wyznacza pozycję i orientację nadajnika, uwzględniając przy tym 
 jego kształt (nadmiarowość danych), a także prawdopodobieństwo, że znaleziony \textit{wzorzec} znajduje się w określonej pozycji.
 Rysunek \ref{fig:blad_korel} przedstawia wynik przetwarzania sygnału przez moduł \textit{find\_pattern.py}.

\begin{figure}[h!]
    \centering
    \includegraphics[width=1.13\textwidth, trim= 53mm 0mm 0mm 0mm,clip]{blad_korel}
    \caption{Sygnał przetworzony przez moduł \textit{find\_pattern.py}}
    \label{fig:blad_korel}
\end{figure}
 
 
 Analiza złożoności obliczeniowej: wyliczenie zarówno $ \sum\limits_{t=0}^{n-1}  w^2(t) $,
jak i $\sum\limits_{t=0}^{n-1} f^2(t+x)$ wymaga jedynie liniowej liczby operacji, a 
 $\sum\limits_{t=0}^{n-1}  w(t) f(t+x) $ jest korelacją wzajemną funkcji $w(t)$ i $f(x)$, którą
 można wyliczyć w czasie $m \log(m)$, korzystając z szybkiej transformacji Fouriera \cite{bib:FFT_correlation},
 gdzie $m = \max \{|w|, |f| \}$. Wyliczenie całej funkcji $e(x)$ wymaga zatem jedynie $m \log(m)$ operacji.

 
 Kolejnym zadaniem modułu \textit{find\_pattern.py} jest uaktualnianie \textit{wzorca}.
 Wraz ze zmianą kąta nachylenia nadajnika względem odbiornika zmienia się kształt odbieranego sygnału,
 dlatego jeśli badany sygnał zawiera szukany \textit{wzorzec} oraz spełnione są następujące warunki:
 \begin{itemize}
  \item moc sygnału pasującego do \textit{wzorca} jest wystarczająco duża,
  \item błąd względny pomiędzy sygnałem a \textit{wzorcem} jest mały,
  \item moduł \textit{xyz.py} rozpozna otrzymane pozycje jako pasuje do modelowanej rzeczywistości
 \end{itemize}
to moduł uaktualni \textit{wzorzec}.
 
 
\section{Wyznaczanie pozycji oraz orientacji przestrzennej, moduł \textit{xyz.py}}

Na ostatnim etapem obliczeń wyznaczane są współrzędne kartezjańskie czterech głośników umieszczonych na nadajniku.
Zajmuje się tym moduł \textit{xyz.py}, który dodatkowo zawiera prosty model rzeczywistości
wyłapujący niepoprawne dane o położeniu głośników.
Niepoprawne dane powstają, gdy część głośników traci widoczność z mikrofonami, np. gdy zostaną przysłonięte.
W ramach modelowanej rzeczywistości określa się:
\begin{itemize}
  \item prędkość, z jaką porusza się każdy z głośników,
  \item kształt nadajnika -- położenie głośników względem siebie w przestrzeni.
\end{itemize}
Jeśli któreś z wyliczeń odbiega od ustalonej normy, to dany pomiar uznawany jest za błędny.


Współrzędne głośników wyznacza się na podstawie dwunastu funkcji $e_{ij}(x)$ dla $i=1, ..., 4$ oraz $j=1, ..., 3$,
w których zawarta jest informacja o odległości pomiędzy głośnikiem $i$ oraz mikrofonem $j$.
Dla każdej funkcji $e_{ij}(x)$ wyszukiwane są pozycje minimów lokalnych, 
z których następnie wybierane jest $t$ pozycji $x^{ij}_1, ..., x^{ij}_t$
o najmniejszych wartościach, gdzie $t$ jest stałą wewnętrzną modułu \textit{xyz.py}.
Pozycje $x^{ij}_k$ dla $k=1, ..., t$  można interpretować jako najbardziej prawdopodobne odległości pomiędzy głośnikiem, a mikrofonem.

Do wyliczenia położenia głośników wykorzystywane są wszystkie możliwe kombinacje wyznaczonych odległości.
Zauważmy, że jest ich $12^t$. Każda z tych kombinacji 
sprawdza się czy pasuje do modelowanej rzeczywistości. Z tych, które są akceptowalne, wybierana jest 
kombinacja o najmniejszym sumarycznym błędzie średniokwadratowym, czyli najbardziej prawdopodobna.

Ostatecznym wynikiem pracy modułu \textit{xyz.py} są wartości $x^{ij}$, czyli najbardziej prawdopodobne odległości
pomiędzy  $i$-tym głośnikiem (gdzie $i=1,...,4$), a $j$-tym mikrofonem, na podstawie których 
wyznacza się  pozycję oraz orientację nadajnika.

Wartości $x^{ij}$ przekazywane są również z powrotem do modułu \textit{find\_pattern.py} w celu aktualizacji \textit{wzorców}.


\section{Kalibracja}

Przed każdym uruchomieniem urządzenie wymagana kalibracji,
która sprowadza się do ustawienia nadajnika w odległości 2 metrów od odbiornika,
uruchomienia programu \textit{save-pattern.py} oraz zaznaczenia  12 obszarów  będącym czołem odebranego sygnału.
Obszary te wykorzystane zostaną jako \textit{wzorce} podczas mierzenia odległości.
Program wyliczy także aktualną prędkość dźwięku, uwzględniając odległość nadajnika od odbiornika.

Rysunek \ref{fig:kalibracja_12x} przedstawia 12 sygnałów wraz z zaznaczonymi wzorcami.


 \begin{figure}[h!]
    \centering
    \includegraphics[width=1.12\textwidth, trim= 46mm 0mm 0mm 0mm,clip]{kalibracja_12x}
    \caption{Kalibracja, 12 sygnałów z zaznaczonymi \textit{wzorcami}}
    \label{fig:kalibracja_12x}
\end{figure}

\newpage

\section{Obsługa programu \textit{scan.py}}

Do obsługi prototypu służy program \textit{scan.py}. Po jego uruchomieniu 
na monitorze wyświetlają się trzy okna: 
\begin{itemize}
 \item widok 3D prezentujący położenie nadajnika w przestrzeni -- w oknie widoczne są  
 cztery punkty reprezentujące głośniki nadajnika oraz trzy prostopadłe względem siebie wektory określające 
jego orientację i położenie (rysunek \ref{fig:widok3d}),
 \item widok 2D, również przestawiające położenie nadajnika w przestrzeni --
 są to nałożone na siebie rzuty na płaszczyzny XY i XZ
 (rysunek \ref{fig:widok2d}),
 \item diagram informujący o mocy całego odebranego sygnału, mocy \textit{wzorca}, mocy sygnału pasującego do \textit{wzorca} 
 oraz błędzie pomiędzy \textit{wzorcem} a odebranym sygnałem (rysunek \ref{fig:power}).
\end{itemize}

Wszelkie zmiany położenia nadajnika automatycznie odświeżają wszystkie widoki. 
Podczas uruchamiania programu aktualne położenie nadajnika przyjmowane 
jest jako początek układu współrzędnych.
Jeśli któryś z głośników straci widoczność z mikrofonami, obraz przestaje się odświeżać, a informacja, dlaczego tak
się dzieje, widoczna jest w oknie przedstawiającym moc sygnału.


\rysunek{3d}{Widok 3D}{\label{fig:widok3d}}
\rysunek{2d}{Widok 2D}{\label{fig:widok2d}}
\rysunek{power}{Diagram informujący o mocy całego odebranego sygnału, mocy \textit{wzorca}, mocy sygnału pasującego do \textit{wzorca} 
 oraz błędzie pomiędzy \textit{wzorcem} a odebranym sygnałem}{\label{fig:power}}


%\chapter{Kalibracja}\label{r:kalibracja}


\chapter{Podsumowanie}
\section{Zalety i wady}



\appendix
\chapter{Załączniki}

Na płycie CD-ROM dołączonej do niniejszej pracy znajdują się:
\begin{itemize}
 \item kopia niniejszej pracy w PDF i jej tekst źródłowy w języku \LaTeX,
 \item kod źródłowy nadajnika,
 \item kod źródłowy odbiornika,
 \item kod źródłowy programu \textit{scan.py} 
 \item kod źródłowy programu do kalibracji \textit{save-pattern.py}
 \item kod źródłowy biblioteki \textit{mp3d}, na którą składają się:
\end{itemize}

  \begin{enumerate}
    \item moduł \textit{mp3d com.py}
    \item moduł \textit{mp3d find_pattern.py}
    \item moduł \textit{mp3d info.py}
    \item moduł \textit{mp3d ply.py}
    \item moduł \textit{mp3d xyz.py}
  \end{enumerate} 

\begin{thebibliography}{99}

\addcontentsline{toc}{chapter}{Bibliografia}

  \bibitem{bib:soundSpeed}  Dean, E. A., \textit{Atmospheric Effects on the Speed of Sound}, BATTELLE COLUMBUS LABS DURHAM NC, AUG 1979
  \myurl{http://www.dtic.mil/cgi-bin/GetTRDoc?Location=U2&doc=GetTRDoc.pdf&AD=ADA076060}

  \bibitem{bib:stm32_usb_101} Marcin Peczarski, 
  \textit{USB dla niewtajemniczonych w przykładach na mikrokontrolery STM32}, Wydawnictwo BTC, Legionowo 2013,
  \myurl{http://www.mimuw.edu.pl/~marpe/book/stm32_usb_101.zip}

  \bibitem{bib:FFT_correlation} Weisstein, Eric W. \textit{Cross-Correlation Theorem.} From MathWorld--A Wolfram Web Resource. 
  \myurl{http://mathworld.wolfram.com/Cross-CorrelationTheorem.html}

  \bibitem{bib:arduinoNano}  
  \textit{Arduino Nano (v2.3) User Manual}
  \myurl{https://www.arduino.cc/en/uploads/Main/ArduinoNanoManual23.pdf}
  
  \bibitem{bib:40ST12}
  \textit{ULTRASONIC SENSOR (GENERAL), 40ST-12}
  \myurl{https://www.maritex.com.pl/media/uploads/PRODUKTY_PDF/sens/40str-12.pdf}

  \bibitem{bib:stm32f4Discovery} Discovery kit for STM32F407/417 lines,
  \myurl{http://www.st.com/web/catalog/tools/FM116/SC959/SS1532/PF252419}

  \bibitem{bib:stm32f407} datasheet: STM32F405xx, STM32F407xx 
  \myurl{http://www.st.com/st-web-ui/static/active/en/resource/technical/document/datasheet/DM00037051.pdf}
  
  \bibitem{bib:wzm_ladunkowy} James Karki, \textit{Signal Conditioning Piezoelectric Sensors}, 
  Application Report: SLOA033A - September 2000
  \myurl{http://www.ti.com/lit/an/sloa033a/sloa033a.pdf}

  \bibitem{bib:ne5532} \textit{Dual low-noise operational amplifiers}. SLOS075I, november 1979, revised april 2009
  \myurl{http://www.ti.com/lit/ds/symlink/ne5532.pdf}

  \bibitem{bib:TLV2774} \textit{TLV277x, TLV277xA - Family of 2.7-V high-slew-rate rail-to-rail output operation amplifiers with shutdown.}
  SLOS209G - january 1998 - revised february 2004
  \myurl{http://www.ti.com/lit/ds/symlink/tlv2774.pdf}

  \bibitem{bib:atmega328} datasheet, \textit{ATmega48A/PA/88A/PA/168A/PA/328/P - 
  ATMEL 8-BIT MICROCONTROLLER WITH 4/8/16/32KBYTES IN-SYSTEM PROGRAMMABLE FLASH}
  Atmel 2014
  \myurl{http://www.atmel.com/images/Atmel-8271-8-bit-AVR-Microcontroller-ATmega48A-48PA-88A-88PA-168A-168PA-328-328P_datasheet_Complete.pdf}

  \bibitem{bib:Arduino} \textit{Arduino Software}
  Arduino
  \myurl{https://www.arduino.cc/en/Main/Software}

  \bibitem{bib:OculusRift}  Parth Rajesh Desai, Pooja Nikhil Desai, Komal Deepak Ajmera, Khushbu Mehta,
  \textit{A Review Paper on Oculus Rift-A Virtual Reality Headset}
  \myurl{http://arxiv.org/pdf/1408.1173.pdf}

  \bibitem{bib:OculusRiftDK2}  vrwiki,
  \textit{Oculus Rift Development Kit 2}
  \myurl{https://vrwiki.wikispaces.com/Oculus+Rift+Development+Kit+2}
  
  \bibitem{bib:castAR} Technical Illusions,
  \textit{Press Kit}
  \myurl{http://technicalillusions.com/wp-content/uploads/2013/10/Press-Kit.pdf}

  \bibitem{bib:MicrosoftKinect} Wenjun Zeng,
  \textit{Microsoft Kinect Sensor and Its Effect}, Multimedia at Work
  \myurl{http://research.microsoft.com/en-us/um/people/zhang/Papers/Microsoft\%20Kinect\%20Sensor\%20and\%20Its\%20Effect\%20-\%20IEEE\%20MM\%202012.pdf}

  \bibitem{bib:VisualSFM} Changchang Wu,
  \textit{VisualSFM : A Visual Structure from Motion System}
  \myurl{http://ccwu.me/vsfm/}

  \bibitem{bib:mouse}  Christian Banker, Michael Cretella, Jeff Dimaria, Jamie Mitchell, Jeff Tucker, 
  \textit{Ultrasonic 3D Wireless Computer Mouse}, 
  Worcester Polytechnic Institute Electrical and Computer Engineering Department 
  \myurl{https://www.wpi.edu/Pubs/E-project/Available/E-project-042607-125024/unrestricted/magicmouse_2007-04-26.pdf}

  \bibitem{bib:pi3dscan}
  \textit{The Pi 3D scanner project}
  \myurl{http://www.pi3dscan.com/}

  \bibitem{bib:katyEulera}
  \textit{Kąty Eulera} wikipedia
  \myurl{https://pl.wikipedia.org/wiki/K\%C4\%85ty_Eulera}
  
  
\end{thebibliography}

%\include{dodatek}



\end{document}


%%% Local Variables:
%%% mode: latex
%%% TeX-master: t
%%% coding: latin-2
%%% End:

