\chapter{Podstawy teoretyczne}
\section{Wyznaczanie położenia na podstawie odległości od stałych punktów}



\section{Pomiar odległości za pomocą ultradźwięków}

Prezentowany prototyp wykorzystuje metodę pomiaru czasu jaki 
potrzebuje sygnał ultradźwiękowy aby pokonać drogę od nadajnika do odbiornika,
rysunek \ref{fig:pomiar_odleglosci} przedstawia poglądowy schemat pomiarowy.
\rysunekW{pomiar_odleglosci}{pomiar odległości za pomocą ultradźwięków}{\label{fig:pomiar_odleglosci}}{0.4}

Znając prędkość rozchodzenia się fal dźwiękowych w powietrzy oraz czas jaki fala dźwiękowa potrzebowała
aby pokonać dystans od odbiornika do nadajnika jesteśmy w stanie w prosty sposób wyznaczyć szukaną odległość.

Niestety prędkość dźwięku w powietrzu jest mocno zależy od panujących warunków atmosferycznych \cite{bib:soundSpeed},  
głównym czynnikiem wpływającym na prędkość dźwięku jest temperatura,
dla gazu doskonałego prędkość ta wyraża się wzorem:
\[
V_{air} = 331.3  \frac{m}{s}  \sqrt{1+\frac{T}{273.15}}
\]
gdzie: $T$ jest temperaturą w stopniach Celsjusza ($^\circ$C).
wzór ten możemy przybliżyć za pomocą rozwinięcia Taylora do:
\[
 V_{air} = (331.3  +  0.606T) \frac{m}{s}
\]

Mimo iż współczynnik temperaturowy jest stosunkowo mały i stanowi jedynie 0.18\% całej prędkości
to przy pomiarach odległości rzędu metrów i rozdzielczości milimetrowych staje się bardzo istotny. 
Dlatego w prezentowanym urządzeniu istotną częścią jest kalibracja wstępna, podczas której
prędkość dźwięku jest na nowo wyznaczana. Efektem ubocznym kalibracji jest pomiar temperatury otoczenia.



